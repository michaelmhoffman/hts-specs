\documentclass[11pt]{article}
\usepackage[T1]{fontenc}

\usepackage[letterpaper,margin=1in]{geometry}

\usepackage{amsmath}
\usepackage{booktabs}
\usepackage{calc}
\usepackage{caption}
\usepackage[flushmargin,hang]{footmisc}
\usepackage{float}
\usepackage{microtype}
\usepackage{newverbs}
\usepackage{tablefootnote}
\usepackage{tabularx}
\usepackage{todonotes}
\usepackage[hyperfootnotes=false]{hyperref} % doesn't work in tabulars as currently set
\usepackage[nohyperlinks]{acronym}
\usepackage{footnotehyper}
\usepackage[strict]{changepage}
\usepackage[binary-units=true]{siunitx}
\usepackage{enumitem}
\usepackage{stackengine}

\documentclass[11pt]{article}
\usepackage[T1]{fontenc}

\usepackage[letterpaper,margin=1in]{geometry}

\usepackage{amsmath}
\usepackage{booktabs}
\usepackage[flushmargin,hang]{footmisc}
\usepackage{microtype}
\usepackage{newverbs}
\usepackage{tablefootnote}
\usepackage{tabularx}
\usepackage{todonotes}
\usepackage[hyperfootnotes=false]{hyperref} % doesn't work in tabulars as currently set
\usepackage[nohyperlinks]{acronym}
\usepackage{footnotehyper}
\usepackage[strict]{changepage}
\usepackage[binary-units=true]{siunitx}

\documentclass[11pt]{article}
\usepackage[T1]{fontenc}

\usepackage[letterpaper,margin=1in]{geometry}

\usepackage{amsmath}
\usepackage{booktabs}
\usepackage[flushmargin,hang]{footmisc}
\usepackage{microtype}
\usepackage{newverbs}
\usepackage{tablefootnote}
\usepackage{tabularx}
\usepackage{todonotes}
\usepackage[hyperfootnotes=false]{hyperref} % doesn't work in tabulars as currently set
\usepackage[nohyperlinks]{acronym}
\usepackage{footnotehyper}
\usepackage[strict]{changepage}
\usepackage[binary-units=true]{siunitx}

\documentclass[11pt]{article}
\usepackage[T1]{fontenc}

\usepackage[letterpaper,margin=1in]{geometry}

\usepackage{amsmath}
\usepackage{booktabs}
\usepackage[flushmargin,hang]{footmisc}
\usepackage{microtype}
\usepackage{newverbs}
\usepackage{tablefootnote}
\usepackage{tabularx}
\usepackage{todonotes}
\usepackage[hyperfootnotes=false]{hyperref} % doesn't work in tabulars as currently set
\usepackage[nohyperlinks]{acronym}
\usepackage{footnotehyper}
\usepackage[strict]{changepage}
\usepackage[binary-units=true]{siunitx}

\input{BEDv1.ver}

\hypersetup{colorlinks=true,
  linkcolor=blue,
  filecolor=magenta,
  urlcolor=blue,
  pdfinfo={githash=\commitdesc}}

\definecolor{cverbbg}{gray}{0.93}

\title{The Browser Extensible Data~(BED) format}
\author{Jeffrey Niu, Danielle Denisko, Michael M.~Hoffman}
\date{\headdate}

\setlength{\emergencystretch}{\hsize}
\setlength{\footnotemargin}{1em}

\interfootnotelinepenalty=1000000
\makesavenoteenv{tabularx}

\newcolumntype{L}{>{\raggedright\arraybackslash}X}

\providecommand*{\Ac}[1]{\ac{#1}} % work around outdated acronym.sty packages

\frenchspacing

% eliminate passive voice warnings
% chktex-file -3

\begin{document}

\maketitle

\begin{small}
\noindent
The master version of this document can be found at \url{https://github.com/samtools/hts-specs}.
This printing is version~\commitdesc\ from that repository, last modified on the date shown above.
\end{small}

\acused{ASCII}

\section{Specification}

\Ac{BED} is a whitespace-delimited file format, where each~\textbf{file} consists of one or more~\textbf{line}s.\footnote{``Frequently Asked Questions: Data File Formats.'' \ac{UCSC} Genome Browser FAQ, \url{https://genome.ucsc.edu/FAQ/FAQformat.html}}
Each~\textbf{line} describes discrete genomic~\textbf{feature}s by physical start and end position on a linear~\textbf{chromosome}.
The file extension for the \ac{BED} format is~\texttt{.bed}.

\subsection{Typographic conventions}

This document uses the following typographic conventions:

\vspace{2ex}

\noindent
\begin{tabularx}{\textwidth}{r L L}
  \toprule
  Style & Meaning & Examples \\
  \midrule
  Bold & Terms defined in subsections~\ref{sec:terms}--\ref{sec:lines} & \textbf{chromosome}{\quad}\textbf{file} \\
  Sans serif & Names of~\textbf{field}s & \textsf{chrom}{\quad}\textsf{chromStart}{\quad}\textsf{chromEnd} \\
  Fixed-width & Literals or \ac{regex}es\footnote{POSIX/IEEE~1003.1--2017 Extended Regular Expressions, for the ``C'' locale.
                \emph{IEEE Standard for Information Technology---Portable Operating System Interface~(POSIX) Base Specifications}, IEEE~1003.1--2017, 2017} & \texttt{.bed}{\quad}\texttt{grep}{\quad}\texttt{[[:alnum:]]+}{\quad}\texttt{ATCG} \\
  \bottomrule
\end{tabularx}

\subsection{Terminology and concepts}\label{sec:terms}
\begin{description}
\item[0-start, half-open coordinate system:]
  A coordinate system where the first base starts at position~0, and the start of the interval is included but the end is not.
  For example, for a sequence of bases~\texttt{ACTGCG}, the bases given by the interval~[2,~4) are~\texttt{TG}. % chktex 9

\item[BED$n$:]
  A~\textbf{file} with the first $n$~\textbf{field}s of the \ac{BED} format.
  For example, \textbf{BED3}~means a~\textbf{file} with only the first 3~\textbf{field}s; \textbf{BED12}~means a~\textbf{file} with all 12~\textbf{field}s.

\item[BED$n$+:]
  A~\textbf{file} that has $n$~\textbf{field}s of the \ac{BED} format, followed by any number of~\textbf{field}s of custom data defined by a user.

\item[BED$n$+$m$:]
  A~\textbf{file} that has a custom tab-delimited format starting with the first $n$~\textbf{field}s of the \ac{BED} format, followed by $m$~\textbf{field}s of custom data defined by a user.
  For example, \textbf{BED6+4}~means a~\textbf{file} with the first 6~\textbf{field}s of the \ac{BED} format, followed by 4~user-defined~\textbf{field}s.

\item[block:]
  Linear subfeatures within a~\textbf{feature}.
  Usually used to designate exons.

\item[chromosome:]
  A sequence of nucleobases with a name.
  In this specification, ``chromosome'' may also describe a named scaffold that does not fit the biological definition of a chromosome.
  Often, chromosomes are numbered starting from~\texttt{1}.
  There are also often sex chromosomes such as~\texttt{W}, \texttt{X}, \texttt{Y}, and~\texttt{Z}, mitochondrial chromosomes such as~\texttt{M}, and possibly scaffolds from an unknown chromosome, often labeled~\texttt{Un}.
  The name of each chromosome is often prefixed with~\texttt{chr}.
  Examples of chromosome names include~\texttt{chr1}, \texttt{21}, \texttt{chrX}, \texttt{chrM}, \texttt{chrUn}, \texttt{chr19\_KI270914v1\_alt}, and~\texttt{chrUn\_KI270435v1}.

\item[feature:]
  A linear region of a~\textbf{chromosome} with specified properties.
  For example, a~\textbf{file}'s~\textbf{feature}s might all be peaks called from ChIP-seq data, or transcript.

\item[field:]
  Data stored as non-tab text.
  All~\textbf{field}s are 7-bit US \ac{ASCII}.

\item[file:]
  Sequence of one or more~\textbf{line}s.

\item[line:]
  String terminated by a~\textbf{line separator}, in one of the following classes.
  Either a~\textbf{data line}, a~\textbf{comment line}, or a~\textbf{blank line}.
  Discussed more fully in~\autoref{sec:lines}

\item[line separator:]
  Either carriage return, line feed, or carriage return followed by line feed.
  The same \textbf{line separator} must be used throughout the \textbf{file}.
\end{description}

\subsection{Lines}\label{sec:lines}

\subsubsection{Data lines}

Data lines contain \textbf{feature}~information.
A data line is composed of~\textbf{field}s separated by whitespace.
The whitespace must match the \ac{regex}~\texttt{[[:space:]]+}\footnote{\texttt{[[:space:]]} includes the following characters: space, form-feed, newline, carriage-return, tab, and vertical-tab}.

\subsubsection{Comment lines and blank lines}

Both comment lines and blank lines provide no~\textbf{feature} data.

Comment lines start with~\texttt{\#} with no whitespace beforehand.
A~\texttt{\#} appearing anywhere else in a line is treated as~\textbf{feature} data, not a comment.

Blank lines consist entirely of whitespace.
Both comment and blank lines may appear as any line in a~\textbf{file}, at the beginning, middle, or end of the file.
They may appear in any quantity.

\subsection{\acs{BED} fields}

Each~\textbf{data line} contains between~3 and 12~whitespace-delimited~\textbf{field}s.
The first 3~\textbf{field}s are mandatory, and the last 9~\textbf{field}s are optional.
In optional~\textbf{field}s, the order is binding---if 1~\textbf{field} is filled, then all previous~\textbf{field}s must also be filled.
However, \textbf{BED10} and \textbf{BED11} are prohibited.

In a \ac{BED}~\textbf{file}, each~\textbf{data line} must have the same number of~\textbf{field}s.
The positions in \ac{BED}~\textbf{field}s are all described in the~\textbf{0-based, half-open coordinate system}.

\begin{adjustwidth}{-0.5in}{-0.5in}
  \noindent
  \begin{tabularx}{\linewidth}{r l l l L}
    \toprule
    Col & Field & Type & Regex or range & Brief description \\
    \midrule
    1 & \textsf{chrom} & String & \texttt{[[:alnum:]\_]\{1,255\}}{\footnotemark} & \textbf{Chromosome} name \\
    2 & \textsf{chromStart} & Int & $[0, 2^{64}-1]$ & \textbf{Feature} start position \\
    3 & \textsf{chromEnd} & Int & $[0, 2^{64} -1]$ & \textbf{Feature} end position \\
    4 & \textsf{name} & String & \texttt{[{\textasciicircum}{\textbackslash}t]\{0,255\}} & \textbf{Feature} description \\
    5 & \textsf{score} & Int & $[0, 1000]$ & A numerical value \\
    6 & \textsf{strand} & String & \texttt{[-+.]} & \textbf{Feature} strand \\
    7 & \textsf{thickStart} & Int & $[0, 2^{64}-1]$ & Thick start position \\
    8 & \textsf{thickEnd} & Int & $[0, 2^{64}-1]$ & Thick end position \\
    9 & \textsf{itemRgb} & Int,Int,Int & \texttt{(}$[0, 255], [0,255], [0,255]$\texttt{) | 0} & Display color \\ % chktex 9
    10 & \textsf{blockCount} & Int & $[0, \textsf{chromEnd}-\textsf{chromStart}]${\footnotemark} & Number of \textbf{block}s \\
    11 & \textsf{blockSizes} & List[Int] & \texttt{([[:digit:]]+,)\{\textsf{blockCount}$-1$\}[[:digit:]]+,?}{\footnotemark} & \textbf{Block} sizes \\
    12 & \textsf{blockStarts} & List[Int] & \texttt{([[:digit:]]+,)\{\textsf{blockCount}$-1$\}[[:digit:]]+,?} & \textbf{Block} start positions \\
    \bottomrule
  \end{tabularx}\label{sec:table}
  \footnotetext[5]{\texttt{[[:alnum:]\_]} is equivalent to the \ac{regex} \texttt{[A-Za-z0-9\_]}. % chktex 8
      It is also equivalent to the Perl extension \texttt{[[:word:]]}.}
  \footnotetext[6]{\textsf{chromEnd}-\textsf{chromStart} is the maximum number of~\textbf{block}s that may exist without overlaps.}
  \footnotetext{For example, if~$\textsf{blockCount} = 4$, then the allowed \ac{regex} would be~\texttt{([[:digit:]]+,)\{3\}[[:digit:]]+,?}}
\end{adjustwidth}

\subsection{Coordinates}
\begin{enumerate}
\item \textsf{chrom}: The name of the~\textbf{chromosome} or scaffold where the~\textbf{feature} is present.
  Limiting only to word characters only, instead of all non-whitespace characters, makes \ac{BED}~\textbf{file}s more portable to varying environments which may make different assumptions about allowed characters.
  The name must be between~1 and 255~characters long, inclusive.

\item \textsf{chromStart}: Start position of the~\textbf{feature} on the~\textbf{chromosome} or scaffold.
  \textsf{chromStart}~must be an integer greater than or equal to~0 and less than the total number of bases of the~\textbf{chromosome} to which it belongs.
  If the size of the~\textbf{chromosome} is unknown, then \textsf{chromStart}~must be less than or equal to~$2^{64} - 1$, which is the maximum size of an unsigned 64-bit integer.

\item \textsf{chromEnd}: End position of the~\textbf{feature} on the~\textbf{chromosome} or scaffold.
  \textsf{chromEnd}~must be an integer greater than or equal to the value of~\textsf{chromStart} and less than or equal to the total number of bases in the~\textbf{chromosome} to which it belongs.
  If the size of the~\textbf{chromosome} is unknown, then \textsf{chromEnd}~must be less than or equal to~$2^{64} - 1$, the maximum size of an unsigned 64-bit integer.
\end{enumerate}

\subsection{Simple attributes}
\begin{enumerate}
  \setcounter{enumi}{3}

\item \textsf{name}: String that describes the~\textbf{feature}.
  The name must be~0 to 255~non-tab characters.
  The name must not be empty or contain whitespace, unless all fields in file are delimited exclusively using single tab characters.
  A visual representation of the \ac{BED} format may display the name next to the~\textbf{feature}.

\item \textsf{score}: Integer between~0 and~1000, inclusive.
  If the~\textbf{feature} has no score information, then~\texttt{0} should be used as a default value.
  A visual representation of the \ac{BED} format may shade features differently depending on their score.

\item \textsf{strand}: Strand that the~\textbf{feature} appears on.
  The strand may either refer to the~\texttt{+}~(sense or coding) strand or the~\texttt{-}~(antisense or complementary) strand.
  If the~\textbf{feature} has no strand information or unknown strand, then a dot~(\texttt{.}) must be used.
\end{enumerate}

\subsection{Display attributes}
\begin{enumerate}
  \setcounter{enumi}{6}

\item \textsf{thickStart}: Start position at which the~\textbf{feature} is visualized with a thicker or accented display.
  This value must be an integer between~\textsf{chromStart} and~\textsf{chromEnd}, inclusive.
  There is no specified default value for~\textsf{thickStart}.

\item \textsf{thickEnd}: End position at which the~\textbf{feature} is visualized with a thicker or accented display.
  This value must be an integer greater than or equal to~\textsf{thickStart} and less than or equal to~\textsf{chromEnd}, inclusive.
  In \ac{BED} files with fewer than 7~\textbf{field}s, the whole~\textbf{feature} has thick display.
  In \textbf{BED7+}~files, to achieve the same effect, set \textsf{thickStart}~equal to~\textsf{chromStart} and \textsf{thickEnd}~equal to~\textsf{chromEnd}.
  If this~\textbf{field} is not specified but \textsf{thickStart}~is, then the entire~\textbf{feature} has thick display.
  There is no specified default value for~\textsf{thickEnd}.

\item \textsf{itemRgb}: A triple of integers that determines the color of this~\textbf{feature} when visualized.
  The triple is three integers separated by commas.
  Each integer is between~0 and~255, inclusive.
  To make a~\textbf{feature} black, \textsf{itemRgb}~may be a single~\texttt{0}, which is visualized identically to a~\textbf{feature} with \textsf{itemRgb} of \texttt{0,0,0}.
\end{enumerate}

\subsection{Blocks}
\begin{enumerate}
  \setcounter{enumi}{9}

\item \textsf{blockCount}: Number of~\textbf{block}s in the~\textbf{feature}.
  \textsf{blockCount}~must be an integer greater than 0.
  \textsf{blockCount}~is mandatory in~\textbf{BED12+}~files.
  Null or empty~\textsf{blockCount} are not allowed, because~\textsf{blockSizes} and~\textsf{blockStarts} rely on~\textsf{blockCount}.
  A visual representation of the \ac{BED} format may have blocks appear thicker than the rest of the~\textbf{feature}.

\item \textsf{blockSizes}: Comma-separated list of length~\textsf{blockCount} containing the size of each~\textbf{block}.
  There must be no spaces before or after commas.
  There may be a trailing comma after the last element of the list.
  \textsf{blockSizes}~is mandatory in \textbf{BED12+} files.
  Null or empty~\textsf{blockSizes} is not allowed, because \textsf{blockStarts}~cannot be verified without~\textsf{blockSizes}.

\item \textsf{blockStarts}: Comma-separated list of length~\textsf{blockCount} containing each \textbf{block}'s~start position, relative to~\textsf{chromStart}.
  There must not be spaces before or after the commas.
  There may be a trailing comma after the last element of the list.
  Each element in~\textsf{blockStarts} is paired with the corresponding element in~\textsf{blockSizes}.
  Each \textsf{blockStarts}~element must be an integer between~0 and~$\textsf{chromEnd} - \textsf{chromStart}$, inclusive.
  For each couple~$i$ of~$(\textsf{blockStarts}_i, \textsf{blockSizes}_i)$, the quantity~$\textsf{chromStart} + \textsf{blockStarts}_i + \textsf{blockSizes}_i$ must be less or equal to \textsf{chromEnd}.
  These conditions enforce that each~\textbf{block} is contained within the~\textbf{feature}.
  The first~\textbf{block} must start at~\textsf{chromStart} and the last~\textbf{block} must end at~\textsf{chromEnd}.
  Moreover, the~\textbf{block}s must not overlap.
  The list must be sorted in ascending order.
  \textsf{blockStarts}~is mandatory in~\textbf{BED12+} files.
  Null or empty~\textsf{blockStarts} is not allowed.
\end{enumerate}

\section{Examples}

\subsection[title]{Example BED6 file from the \acs{UCSC} Genome Browser FAQ\footnote{``Frequently
    Asked Questions: Data File Formats.'' \ac{UCSC} Genome Browser FAQ,
    \url{https://genome.ucsc.edu/FAQ/FAQformat.html}}}\label{sec:example-bed6}

\begin{verbatim}
chr7  127471196  127472363  Pos1  0  +
chr7  127472363  127473530  Pos2  0  +
chr7  127473530  127474697  Pos3  0  +
chr7  127474697  127475864  Pos4  0  +
chr7  127475864  127477031  Neg1  0  -
chr7  127477031  127478198  Neg2  0  -
chr7  127478198  127479365  Neg3  0  -
chr7  127479365  127480532  Pos5  0  +
chr7  127480532  127481699  Neg4  0  -
\end{verbatim}

\subsection{Example BED12 file from the \acs{UCSC} Genome Browser FAQ}
\begin{verbatim}
chr22 1000 5000 cloneA 960 + 1000 5000 0 2 567,488, 0,3512
chr22 2000 6000 cloneB 900 - 2000 6000 0 2 433,399, 0,3601
\end{verbatim}

The~\textbf{block}s in this example satisfy the required constraints.
The first~\textbf{block} starts at~\textsf{chromStart} since the first~\textsf{blockStarts} element is~0.
The last~\textbf{block} ends at~\textsf{chromEnd} since the last~\textbf{block} starts at position 4512~(1000+3512) with size~488, and therefore ends at position 5000~(4512+488).

\section{Recommended practice for the \acs{BED} format}

\subsection{Mandatory fields}
\begin{itemize}
\item \textsf{chrom}: The name of each~\textbf{chromosome} should also match the names from a reference genome, if applicable.
  For example, in the human genome, the chromosomes may be named~\texttt{chr1} to \texttt{chr22}, \texttt{chrX}, \texttt{chrY}, and~\texttt{chrM}.
  Names should be consistent within a~\textbf{file}.
  For example, one should not use both~\texttt{17} and~\texttt{chr17} to represent the same~\textbf{chromosome} in the same~\textbf{file}.
\end{itemize}

\subsection{Optional fields}\label{sec:optional}
\begin{itemize}
\item \textsf{name}: If a feature has no name, then a dot~(\texttt{.}) should be used.
  Names should avoid using the space character even if the file is exclusively delimited with single tab characters because parsers may interpret a space as a delimiter.

\item \textsf{itemRgb}: Eight or fewer colors should be used as too many colors may slow down visualizations and are difficult for humans to distinguish.\footnote{``Frequently
    Asked Questions: Data File Formats.'' \ac{UCSC} Genome Browser FAQ,
    \url{https://genome.ucsc.edu/FAQ/FAQformat.html}}

\end{itemize}

\subsection{User-defined fields}

Custom data \textbf{fields} may contain any non-tab 7-bit US \ac{ASCII} character.
Definitions of a custom \ac{BED} format should restrict the type of each \textbf{field} to the extent possible.
Each custom \textbf{field} should contain either one of the following data types or a comma-separated list of values of the same type:

\noindent
\begin{tabularx}{\textwidth}{r L}
  \toprule
  Type & Definition \\
  \midrule
  Integer & String representation of 64-bit signed integer\footnote{\emph{IEEE 754--1985 IEEE Standard for Binary Floating-Point Arithmetic.} IEEE 754--1985, 1985} \\
  Unsigned & String representation of 64-bit unsigned integer\footnotemark[10] \\
  Float & String representation of 64-bit floating point number\footnotemark[10] \\
  Character & One character, other than tab \\
  String & One or more characters, other than tab \\
  \bottomrule
\end{tabularx}

This specification does not contain a means for interchanging custom \ac{BED} format definitions.
The AutoSQL format\footnote{Kent, W.~James.
  (2000) ``AutoSQL.''
  \url{https://hgwdev.gi.ucsc.edu/~kent/exe/doc/autoSql.doc}} provides one method for defining custom \ac{BED} formats in a separate file.

\subsection{Sorting}
\Ac{BED} \textbf{file}s should be sorted by~\textsf{chrom}, then by~\textsf{chromStart} numerically, and finally by~\textsf{chromEnd} numerically.
\textsf{chrom} may be sorted using any scheme (such as lexicographic or numeric order), but all lines with the same~\textsf{chrom} value should occur consecutively.
For example, the lexicographic order of~\texttt{chr1}, \texttt{chr10}, \texttt{chr11}, \texttt{chr12}, {\ldots}, \texttt{chr2}, \texttt{chr20}, \texttt{chr21}, {\ldots}, \texttt{chr3}, {\ldots}, \texttt{chrX}, \texttt{chrY}, \texttt{chrM} is an acceptable sorting.
The numeric order of~\texttt{chr1}, \texttt{chr2}, {\ldots}, \texttt{chr21}, \texttt{chr22}, \texttt{chrM}, \texttt{chrX}, \texttt{chrY} is also acceptable.
Regardless of the chromosome sorting scheme, lines for two features on the same chromosome should not have any lines for features on other chromosomes between them.

\subsection{Whitespace}\label{sec:whitespace}
Though lines may use any kind of whitespace as a delimiter between~\textbf{field}s, a single tab~(\texttt{{\textbackslash}t}) should be used.
This is because almost all tools support tabs while some tools do not support other kinds of whitespace.
Also, whitespace within the~\textsf{name}~\textbf{field} may be used only if the \textbf{field}~delimiter is tab throughout the \textbf{file}.

\subsection{Large \acs{BED} files}
If a~\textbf{file} intended for visualization is over \SI{50}{\mebi\byte} in size, the~\textbf{file} should be converted to~\texttt{bigBed} format, which is an indexed binary format.\footnote{Kent, W.~James et al.
  (2010) ``BigWig and BigBed: enabling browsing of large distributed datasets.''
  \emph{Bioinformatics} 26(17):2204--2207.
  \url{https://doi.org/10.1093/bioinformatics/btq351}}
The~\texttt{bedToBigBed} program may perform this conversion.\footnote{``bigBed Track Format.''
  \Ac{UCSC} Genome Browser FAQ, \url{https://genome.ucsc.edu/goldenPath/help/bigBed.html}}

\section{\acs{UCSC} track files}

Track files are files that contain additional information intended for a visualization tool such as the \ac{UCSC} Genome Browser.\footnote{Haeussler, Maximilian et al.
  (2019) ``The \acl{UCSC} Genome Browser database: 2019 update.''
  \emph{Nucleic Acids Research} 47(D1):D853--D858.
  \url{https://doi.org/10.1093/nar/gky1095}}
Track files contain browser lines and track lines that precede lines from a file format supported by the Genome Browser.\footnote{``Displaying your own annotations in the Genome Browser.'' \ac{UCSC} Genome Browser FAQ, \url{https://genome.ucsc.edu/goldenPath/help/customTrack.html\#lines}}
Track files are not valid \ac{BED} files --- valid \ac{BED} files must not have any browser or track lines.
To distinguish between \ac{BED} files and track files, track files should use the file extension~\texttt{.track}.

\section{Acronyms}
\begin{acronym}[ASCII]
  \acro{ASCII}{American Standard Code for Information Interchange}
  \acro{BED}{Browser Extensible Data}
  \acro{GA4GH}{Global Alliance for Genomics and Health}
  \acro{regex}{regular expression}
  \acro{UCSC}{University of California, Santa Cruz}
\end{acronym}

\section{Acknowledgments}

We thank W.~James Kent and the \ac{UCSC} Genome Browser team for creating the \ac{BED} format.
We thank W.~James Kent and Hiram Clawson~(\ac{UCSC}); Eric Roberts~(Princess Margaret Cancer Centre); John Marshall~(University of Glasgow); Aaron R.~Quinlan and Brent S.~Pedersen~(University of Utah); Ting Wang~(Washington University in St.~Louis); and the \ac{GA4GH} File Formats Task Team for comments on this specification.

\end{document}

% chktex-file 17

%%% Local Variables:
%%% mode: latex
%%% TeX-master: t
%%% End:


\hypersetup{colorlinks=true,
  linkcolor=blue,
  filecolor=magenta,
  urlcolor=blue,
  pdfinfo={githash=\commitdesc}}

\definecolor{cverbbg}{gray}{0.93}

\title{The Browser Extensible Data~(BED) format}
\author{Jeffrey Niu, Danielle Denisko, Michael M.~Hoffman}
\date{\headdate}

\setlength{\emergencystretch}{\hsize}
\setlength{\footnotemargin}{1em}

\interfootnotelinepenalty=1000000
\makesavenoteenv{tabularx}

\newcolumntype{L}{>{\raggedright\arraybackslash}X}

\providecommand*{\Ac}[1]{\ac{#1}} % work around outdated acronym.sty packages

\frenchspacing

% eliminate passive voice warnings
% chktex-file -3

\begin{document}

\maketitle

\begin{small}
\noindent
The master version of this document can be found at \url{https://github.com/samtools/hts-specs}.
This printing is version~\commitdesc\ from that repository, last modified on the date shown above.
\end{small}

\acused{ASCII}

\section{Specification}

\Ac{BED} is a whitespace-delimited file format, where each~\textbf{file} consists of one or more~\textbf{line}s.\footnote{``Frequently Asked Questions: Data File Formats.'' \ac{UCSC} Genome Browser FAQ, \url{https://genome.ucsc.edu/FAQ/FAQformat.html}}
Each~\textbf{line} describes discrete genomic~\textbf{feature}s by physical start and end position on a linear~\textbf{chromosome}.
The file extension for the \ac{BED} format is~\texttt{.bed}.

\subsection{Typographic conventions}

This document uses the following typographic conventions:

\vspace{2ex}

\noindent
\begin{tabularx}{\textwidth}{r L L}
  \toprule
  Style & Meaning & Examples \\
  \midrule
  Bold & Terms defined in subsections~\ref{sec:terms}--\ref{sec:lines} & \textbf{chromosome}{\quad}\textbf{file} \\
  Sans serif & Names of~\textbf{field}s & \textsf{chrom}{\quad}\textsf{chromStart}{\quad}\textsf{chromEnd} \\
  Fixed-width & Literals or \ac{regex}es\footnote{POSIX/IEEE~1003.1--2017 Extended Regular Expressions, for the ``C'' locale.
                \emph{IEEE Standard for Information Technology---Portable Operating System Interface~(POSIX) Base Specifications}, IEEE~1003.1--2017, 2017} & \texttt{.bed}{\quad}\texttt{grep}{\quad}\texttt{[[:alnum:]]+}{\quad}\texttt{ATCG} \\
  \bottomrule
\end{tabularx}

\subsection{Terminology and concepts}\label{sec:terms}
\begin{description}
\item[0-start, half-open coordinate system:]
  A coordinate system where the first base starts at position~0, and the start of the interval is included but the end is not.
  For example, for a sequence of bases~\texttt{ACTGCG}, the bases given by the interval~[2,~4) are~\texttt{TG}. % chktex 9

\item[BED$n$:]
  A~\textbf{file} with the first $n$~\textbf{field}s of the \ac{BED} format.
  For example, \textbf{BED3}~means a~\textbf{file} with only the first 3~\textbf{field}s; \textbf{BED12}~means a~\textbf{file} with all 12~\textbf{field}s.

\item[BED$n$+:]
  A~\textbf{file} that has $n$~\textbf{field}s of the \ac{BED} format, followed by any number of~\textbf{field}s of custom data defined by a user.

\item[BED$n$+$m$:]
  A~\textbf{file} that has a custom tab-delimited format starting with the first $n$~\textbf{field}s of the \ac{BED} format, followed by $m$~\textbf{field}s of custom data defined by a user.
  For example, \textbf{BED6+4}~means a~\textbf{file} with the first 6~\textbf{field}s of the \ac{BED} format, followed by 4~user-defined~\textbf{field}s.

\item[block:]
  Linear subfeatures within a~\textbf{feature}.
  Usually used to designate exons.

\item[chromosome:]
  A sequence of nucleobases with a name.
  In this specification, ``chromosome'' may also describe a named scaffold that does not fit the biological definition of a chromosome.
  Often, chromosomes are numbered starting from~\texttt{1}.
  There are also often sex chromosomes such as~\texttt{W}, \texttt{X}, \texttt{Y}, and~\texttt{Z}, mitochondrial chromosomes such as~\texttt{M}, and possibly scaffolds from an unknown chromosome, often labeled~\texttt{Un}.
  The name of each chromosome is often prefixed with~\texttt{chr}.
  Examples of chromosome names include~\texttt{chr1}, \texttt{21}, \texttt{chrX}, \texttt{chrM}, \texttt{chrUn}, \texttt{chr19\_KI270914v1\_alt}, and~\texttt{chrUn\_KI270435v1}.

\item[feature:]
  A linear region of a~\textbf{chromosome} with specified properties.
  For example, a~\textbf{file}'s~\textbf{feature}s might all be peaks called from ChIP-seq data, or transcript.

\item[field:]
  Data stored as non-tab text.
  All~\textbf{field}s are 7-bit US \ac{ASCII}.

\item[file:]
  Sequence of one or more~\textbf{line}s.

\item[line:]
  String terminated by a~\textbf{line separator}, in one of the following classes.
  Either a~\textbf{data line}, a~\textbf{comment line}, or a~\textbf{blank line}.
  Discussed more fully in~\autoref{sec:lines}

\item[line separator:]
  Either carriage return, line feed, or carriage return followed by line feed.
  The same \textbf{line separator} must be used throughout the \textbf{file}.
\end{description}

\subsection{Lines}\label{sec:lines}

\subsubsection{Data lines}

Data lines contain \textbf{feature}~information.
A data line is composed of~\textbf{field}s separated by whitespace.
The whitespace must match the \ac{regex}~\texttt{[[:space:]]+}\footnote{\texttt{[[:space:]]} includes the following characters: space, form-feed, newline, carriage-return, tab, and vertical-tab}.

\subsubsection{Comment lines and blank lines}

Both comment lines and blank lines provide no~\textbf{feature} data.

Comment lines start with~\texttt{\#} with no whitespace beforehand.
A~\texttt{\#} appearing anywhere else in a line is treated as~\textbf{feature} data, not a comment.

Blank lines consist entirely of whitespace.
Both comment and blank lines may appear as any line in a~\textbf{file}, at the beginning, middle, or end of the file.
They may appear in any quantity.

\subsection{\acs{BED} fields}

Each~\textbf{data line} contains between~3 and 12~whitespace-delimited~\textbf{field}s.
The first 3~\textbf{field}s are mandatory, and the last 9~\textbf{field}s are optional.
In optional~\textbf{field}s, the order is binding---if 1~\textbf{field} is filled, then all previous~\textbf{field}s must also be filled.
However, \textbf{BED10} and \textbf{BED11} are prohibited.

In a \ac{BED}~\textbf{file}, each~\textbf{data line} must have the same number of~\textbf{field}s.
The positions in \ac{BED}~\textbf{field}s are all described in the~\textbf{0-based, half-open coordinate system}.

\begin{adjustwidth}{-0.5in}{-0.5in}
  \noindent
  \begin{tabularx}{\linewidth}{r l l l L}
    \toprule
    Col & Field & Type & Regex or range & Brief description \\
    \midrule
    1 & \textsf{chrom} & String & \texttt{[[:alnum:]\_]\{1,255\}}{\footnotemark} & \textbf{Chromosome} name \\
    2 & \textsf{chromStart} & Int & $[0, 2^{64}-1]$ & \textbf{Feature} start position \\
    3 & \textsf{chromEnd} & Int & $[0, 2^{64} -1]$ & \textbf{Feature} end position \\
    4 & \textsf{name} & String & \texttt{[{\textasciicircum}{\textbackslash}t]\{0,255\}} & \textbf{Feature} description \\
    5 & \textsf{score} & Int & $[0, 1000]$ & A numerical value \\
    6 & \textsf{strand} & String & \texttt{[-+.]} & \textbf{Feature} strand \\
    7 & \textsf{thickStart} & Int & $[0, 2^{64}-1]$ & Thick start position \\
    8 & \textsf{thickEnd} & Int & $[0, 2^{64}-1]$ & Thick end position \\
    9 & \textsf{itemRgb} & Int,Int,Int & \texttt{(}$[0, 255], [0,255], [0,255]$\texttt{) | 0} & Display color \\ % chktex 9
    10 & \textsf{blockCount} & Int & $[0, \textsf{chromEnd}-\textsf{chromStart}]${\footnotemark} & Number of \textbf{block}s \\
    11 & \textsf{blockSizes} & List[Int] & \texttt{([[:digit:]]+,)\{\textsf{blockCount}$-1$\}[[:digit:]]+,?}{\footnotemark} & \textbf{Block} sizes \\
    12 & \textsf{blockStarts} & List[Int] & \texttt{([[:digit:]]+,)\{\textsf{blockCount}$-1$\}[[:digit:]]+,?} & \textbf{Block} start positions \\
    \bottomrule
  \end{tabularx}\label{sec:table}
  \footnotetext[5]{\texttt{[[:alnum:]\_]} is equivalent to the \ac{regex} \texttt{[A-Za-z0-9\_]}. % chktex 8
      It is also equivalent to the Perl extension \texttt{[[:word:]]}.}
  \footnotetext[6]{\textsf{chromEnd}-\textsf{chromStart} is the maximum number of~\textbf{block}s that may exist without overlaps.}
  \footnotetext{For example, if~$\textsf{blockCount} = 4$, then the allowed \ac{regex} would be~\texttt{([[:digit:]]+,)\{3\}[[:digit:]]+,?}}
\end{adjustwidth}

\subsection{Coordinates}
\begin{enumerate}
\item \textsf{chrom}: The name of the~\textbf{chromosome} or scaffold where the~\textbf{feature} is present.
  Limiting only to word characters only, instead of all non-whitespace characters, makes \ac{BED}~\textbf{file}s more portable to varying environments which may make different assumptions about allowed characters.
  The name must be between~1 and 255~characters long, inclusive.

\item \textsf{chromStart}: Start position of the~\textbf{feature} on the~\textbf{chromosome} or scaffold.
  \textsf{chromStart}~must be an integer greater than or equal to~0 and less than the total number of bases of the~\textbf{chromosome} to which it belongs.
  If the size of the~\textbf{chromosome} is unknown, then \textsf{chromStart}~must be less than or equal to~$2^{64} - 1$, which is the maximum size of an unsigned 64-bit integer.

\item \textsf{chromEnd}: End position of the~\textbf{feature} on the~\textbf{chromosome} or scaffold.
  \textsf{chromEnd}~must be an integer greater than or equal to the value of~\textsf{chromStart} and less than or equal to the total number of bases in the~\textbf{chromosome} to which it belongs.
  If the size of the~\textbf{chromosome} is unknown, then \textsf{chromEnd}~must be less than or equal to~$2^{64} - 1$, the maximum size of an unsigned 64-bit integer.
\end{enumerate}

\subsection{Simple attributes}
\begin{enumerate}
  \setcounter{enumi}{3}

\item \textsf{name}: String that describes the~\textbf{feature}.
  The name must be~0 to 255~non-tab characters.
  The name must not be empty or contain whitespace, unless all fields in file are delimited exclusively using single tab characters.
  A visual representation of the \ac{BED} format may display the name next to the~\textbf{feature}.

\item \textsf{score}: Integer between~0 and~1000, inclusive.
  If the~\textbf{feature} has no score information, then~\texttt{0} should be used as a default value.
  A visual representation of the \ac{BED} format may shade features differently depending on their score.

\item \textsf{strand}: Strand that the~\textbf{feature} appears on.
  The strand may either refer to the~\texttt{+}~(sense or coding) strand or the~\texttt{-}~(antisense or complementary) strand.
  If the~\textbf{feature} has no strand information or unknown strand, then a dot~(\texttt{.}) must be used.
\end{enumerate}

\subsection{Display attributes}
\begin{enumerate}
  \setcounter{enumi}{6}

\item \textsf{thickStart}: Start position at which the~\textbf{feature} is visualized with a thicker or accented display.
  This value must be an integer between~\textsf{chromStart} and~\textsf{chromEnd}, inclusive.
  There is no specified default value for~\textsf{thickStart}.

\item \textsf{thickEnd}: End position at which the~\textbf{feature} is visualized with a thicker or accented display.
  This value must be an integer greater than or equal to~\textsf{thickStart} and less than or equal to~\textsf{chromEnd}, inclusive.
  In \ac{BED} files with fewer than 7~\textbf{field}s, the whole~\textbf{feature} has thick display.
  In \textbf{BED7+}~files, to achieve the same effect, set \textsf{thickStart}~equal to~\textsf{chromStart} and \textsf{thickEnd}~equal to~\textsf{chromEnd}.
  If this~\textbf{field} is not specified but \textsf{thickStart}~is, then the entire~\textbf{feature} has thick display.
  There is no specified default value for~\textsf{thickEnd}.

\item \textsf{itemRgb}: A triple of integers that determines the color of this~\textbf{feature} when visualized.
  The triple is three integers separated by commas.
  Each integer is between~0 and~255, inclusive.
  To make a~\textbf{feature} black, \textsf{itemRgb}~may be a single~\texttt{0}, which is visualized identically to a~\textbf{feature} with \textsf{itemRgb} of \texttt{0,0,0}.
\end{enumerate}

\subsection{Blocks}
\begin{enumerate}
  \setcounter{enumi}{9}

\item \textsf{blockCount}: Number of~\textbf{block}s in the~\textbf{feature}.
  \textsf{blockCount}~must be an integer greater than 0.
  \textsf{blockCount}~is mandatory in~\textbf{BED12+}~files.
  Null or empty~\textsf{blockCount} are not allowed, because~\textsf{blockSizes} and~\textsf{blockStarts} rely on~\textsf{blockCount}.
  A visual representation of the \ac{BED} format may have blocks appear thicker than the rest of the~\textbf{feature}.

\item \textsf{blockSizes}: Comma-separated list of length~\textsf{blockCount} containing the size of each~\textbf{block}.
  There must be no spaces before or after commas.
  There may be a trailing comma after the last element of the list.
  \textsf{blockSizes}~is mandatory in \textbf{BED12+} files.
  Null or empty~\textsf{blockSizes} is not allowed, because \textsf{blockStarts}~cannot be verified without~\textsf{blockSizes}.

\item \textsf{blockStarts}: Comma-separated list of length~\textsf{blockCount} containing each \textbf{block}'s~start position, relative to~\textsf{chromStart}.
  There must not be spaces before or after the commas.
  There may be a trailing comma after the last element of the list.
  Each element in~\textsf{blockStarts} is paired with the corresponding element in~\textsf{blockSizes}.
  Each \textsf{blockStarts}~element must be an integer between~0 and~$\textsf{chromEnd} - \textsf{chromStart}$, inclusive.
  For each couple~$i$ of~$(\textsf{blockStarts}_i, \textsf{blockSizes}_i)$, the quantity~$\textsf{chromStart} + \textsf{blockStarts}_i + \textsf{blockSizes}_i$ must be less or equal to \textsf{chromEnd}.
  These conditions enforce that each~\textbf{block} is contained within the~\textbf{feature}.
  The first~\textbf{block} must start at~\textsf{chromStart} and the last~\textbf{block} must end at~\textsf{chromEnd}.
  Moreover, the~\textbf{block}s must not overlap.
  The list must be sorted in ascending order.
  \textsf{blockStarts}~is mandatory in~\textbf{BED12+} files.
  Null or empty~\textsf{blockStarts} is not allowed.
\end{enumerate}

\section{Examples}

\subsection[title]{Example BED6 file from the \acs{UCSC} Genome Browser FAQ\footnote{``Frequently
    Asked Questions: Data File Formats.'' \ac{UCSC} Genome Browser FAQ,
    \url{https://genome.ucsc.edu/FAQ/FAQformat.html}}}\label{sec:example-bed6}

\begin{verbatim}
chr7  127471196  127472363  Pos1  0  +
chr7  127472363  127473530  Pos2  0  +
chr7  127473530  127474697  Pos3  0  +
chr7  127474697  127475864  Pos4  0  +
chr7  127475864  127477031  Neg1  0  -
chr7  127477031  127478198  Neg2  0  -
chr7  127478198  127479365  Neg3  0  -
chr7  127479365  127480532  Pos5  0  +
chr7  127480532  127481699  Neg4  0  -
\end{verbatim}

\subsection{Example BED12 file from the \acs{UCSC} Genome Browser FAQ}
\begin{verbatim}
chr22 1000 5000 cloneA 960 + 1000 5000 0 2 567,488, 0,3512
chr22 2000 6000 cloneB 900 - 2000 6000 0 2 433,399, 0,3601
\end{verbatim}

The~\textbf{block}s in this example satisfy the required constraints.
The first~\textbf{block} starts at~\textsf{chromStart} since the first~\textsf{blockStarts} element is~0.
The last~\textbf{block} ends at~\textsf{chromEnd} since the last~\textbf{block} starts at position 4512~(1000+3512) with size~488, and therefore ends at position 5000~(4512+488).

\section{Recommended practice for the \acs{BED} format}

\subsection{Mandatory fields}
\begin{itemize}
\item \textsf{chrom}: The name of each~\textbf{chromosome} should also match the names from a reference genome, if applicable.
  For example, in the human genome, the chromosomes may be named~\texttt{chr1} to \texttt{chr22}, \texttt{chrX}, \texttt{chrY}, and~\texttt{chrM}.
  Names should be consistent within a~\textbf{file}.
  For example, one should not use both~\texttt{17} and~\texttt{chr17} to represent the same~\textbf{chromosome} in the same~\textbf{file}.
\end{itemize}

\subsection{Optional fields}\label{sec:optional}
\begin{itemize}
\item \textsf{name}: If a feature has no name, then a dot~(\texttt{.}) should be used.
  Names should avoid using the space character even if the file is exclusively delimited with single tab characters because parsers may interpret a space as a delimiter.

\item \textsf{itemRgb}: Eight or fewer colors should be used as too many colors may slow down visualizations and are difficult for humans to distinguish.\footnote{``Frequently
    Asked Questions: Data File Formats.'' \ac{UCSC} Genome Browser FAQ,
    \url{https://genome.ucsc.edu/FAQ/FAQformat.html}}

\end{itemize}

\subsection{User-defined fields}

Custom data \textbf{fields} may contain any non-tab 7-bit US \ac{ASCII} character.
Definitions of a custom \ac{BED} format should restrict the type of each \textbf{field} to the extent possible.
Each custom \textbf{field} should contain either one of the following data types or a comma-separated list of values of the same type:

\noindent
\begin{tabularx}{\textwidth}{r L}
  \toprule
  Type & Definition \\
  \midrule
  Integer & String representation of 64-bit signed integer\footnote{\emph{IEEE 754--1985 IEEE Standard for Binary Floating-Point Arithmetic.} IEEE 754--1985, 1985} \\
  Unsigned & String representation of 64-bit unsigned integer\footnotemark[10] \\
  Float & String representation of 64-bit floating point number\footnotemark[10] \\
  Character & One character, other than tab \\
  String & One or more characters, other than tab \\
  \bottomrule
\end{tabularx}

This specification does not contain a means for interchanging custom \ac{BED} format definitions.
The AutoSQL format\footnote{Kent, W.~James.
  (2000) ``AutoSQL.''
  \url{https://hgwdev.gi.ucsc.edu/~kent/exe/doc/autoSql.doc}} provides one method for defining custom \ac{BED} formats in a separate file.

\subsection{Sorting}
\Ac{BED} \textbf{file}s should be sorted by~\textsf{chrom}, then by~\textsf{chromStart} numerically, and finally by~\textsf{chromEnd} numerically.
\textsf{chrom} may be sorted using any scheme (such as lexicographic or numeric order), but all lines with the same~\textsf{chrom} value should occur consecutively.
For example, the lexicographic order of~\texttt{chr1}, \texttt{chr10}, \texttt{chr11}, \texttt{chr12}, {\ldots}, \texttt{chr2}, \texttt{chr20}, \texttt{chr21}, {\ldots}, \texttt{chr3}, {\ldots}, \texttt{chrX}, \texttt{chrY}, \texttt{chrM} is an acceptable sorting.
The numeric order of~\texttt{chr1}, \texttt{chr2}, {\ldots}, \texttt{chr21}, \texttt{chr22}, \texttt{chrM}, \texttt{chrX}, \texttt{chrY} is also acceptable.
Regardless of the chromosome sorting scheme, lines for two features on the same chromosome should not have any lines for features on other chromosomes between them.

\subsection{Whitespace}\label{sec:whitespace}
Though lines may use any kind of whitespace as a delimiter between~\textbf{field}s, a single tab~(\texttt{{\textbackslash}t}) should be used.
This is because almost all tools support tabs while some tools do not support other kinds of whitespace.
Also, whitespace within the~\textsf{name}~\textbf{field} may be used only if the \textbf{field}~delimiter is tab throughout the \textbf{file}.

\subsection{Large \acs{BED} files}
If a~\textbf{file} intended for visualization is over \SI{50}{\mebi\byte} in size, the~\textbf{file} should be converted to~\texttt{bigBed} format, which is an indexed binary format.\footnote{Kent, W.~James et al.
  (2010) ``BigWig and BigBed: enabling browsing of large distributed datasets.''
  \emph{Bioinformatics} 26(17):2204--2207.
  \url{https://doi.org/10.1093/bioinformatics/btq351}}
The~\texttt{bedToBigBed} program may perform this conversion.\footnote{``bigBed Track Format.''
  \Ac{UCSC} Genome Browser FAQ, \url{https://genome.ucsc.edu/goldenPath/help/bigBed.html}}

\section{\acs{UCSC} track files}

Track files are files that contain additional information intended for a visualization tool such as the \ac{UCSC} Genome Browser.\footnote{Haeussler, Maximilian et al.
  (2019) ``The \acl{UCSC} Genome Browser database: 2019 update.''
  \emph{Nucleic Acids Research} 47(D1):D853--D858.
  \url{https://doi.org/10.1093/nar/gky1095}}
Track files contain browser lines and track lines that precede lines from a file format supported by the Genome Browser.\footnote{``Displaying your own annotations in the Genome Browser.'' \ac{UCSC} Genome Browser FAQ, \url{https://genome.ucsc.edu/goldenPath/help/customTrack.html\#lines}}
Track files are not valid \ac{BED} files --- valid \ac{BED} files must not have any browser or track lines.
To distinguish between \ac{BED} files and track files, track files should use the file extension~\texttt{.track}.

\section{Acronyms}
\begin{acronym}[ASCII]
  \acro{ASCII}{American Standard Code for Information Interchange}
  \acro{BED}{Browser Extensible Data}
  \acro{GA4GH}{Global Alliance for Genomics and Health}
  \acro{regex}{regular expression}
  \acro{UCSC}{University of California, Santa Cruz}
\end{acronym}

\section{Acknowledgments}

We thank W.~James Kent and the \ac{UCSC} Genome Browser team for creating the \ac{BED} format.
We thank W.~James Kent and Hiram Clawson~(\ac{UCSC}); Eric Roberts~(Princess Margaret Cancer Centre); John Marshall~(University of Glasgow); Aaron R.~Quinlan and Brent S.~Pedersen~(University of Utah); Ting Wang~(Washington University in St.~Louis); and the \ac{GA4GH} File Formats Task Team for comments on this specification.

\end{document}

% chktex-file 17

%%% Local Variables:
%%% mode: latex
%%% TeX-master: t
%%% End:


\hypersetup{colorlinks=true,
  linkcolor=blue,
  filecolor=magenta,
  urlcolor=blue,
  pdfinfo={githash=\commitdesc}}

\definecolor{cverbbg}{gray}{0.93}

\title{The Browser Extensible Data~(BED) format}
\author{Jeffrey Niu, Danielle Denisko, Michael M.~Hoffman}
\date{\headdate}

\setlength{\emergencystretch}{\hsize}
\setlength{\footnotemargin}{1em}

\interfootnotelinepenalty=1000000
\makesavenoteenv{tabularx}

\newcolumntype{L}{>{\raggedright\arraybackslash}X}

\providecommand*{\Ac}[1]{\ac{#1}} % work around outdated acronym.sty packages

\frenchspacing

% eliminate passive voice warnings
% chktex-file -3

\begin{document}

\maketitle

\begin{small}
\noindent
The master version of this document can be found at \url{https://github.com/samtools/hts-specs}.
This printing is version~\commitdesc\ from that repository, last modified on the date shown above.
\end{small}

\acused{ASCII}

\section{Specification}

\Ac{BED} is a whitespace-delimited file format, where each~\textbf{file} consists of one or more~\textbf{line}s.\footnote{``Frequently Asked Questions: Data File Formats.'' \ac{UCSC} Genome Browser FAQ, \url{https://genome.ucsc.edu/FAQ/FAQformat.html}}
Each~\textbf{line} describes discrete genomic~\textbf{feature}s by physical start and end position on a linear~\textbf{chromosome}.
The file extension for the \ac{BED} format is~\texttt{.bed}.

\subsection{Typographic conventions}

This document uses the following typographic conventions:

\vspace{2ex}

\noindent
\begin{tabularx}{\textwidth}{r L L}
  \toprule
  Style & Meaning & Examples \\
  \midrule
  Bold & Terms defined in subsections~\ref{sec:terms}--\ref{sec:lines} & \textbf{chromosome}{\quad}\textbf{file} \\
  Sans serif & Names of~\textbf{field}s & \textsf{chrom}{\quad}\textsf{chromStart}{\quad}\textsf{chromEnd} \\
  Fixed-width & Literals or \ac{regex}es\footnote{POSIX/IEEE~1003.1--2017 Extended Regular Expressions, for the ``C'' locale.
                \emph{IEEE Standard for Information Technology---Portable Operating System Interface~(POSIX) Base Specifications}, IEEE~1003.1--2017, 2017} & \texttt{.bed}{\quad}\texttt{grep}{\quad}\texttt{[[:alnum:]]+}{\quad}\texttt{ATCG} \\
  \bottomrule
\end{tabularx}

\subsection{Terminology and concepts}\label{sec:terms}
\begin{description}
\item[0-start, half-open coordinate system:]
  A coordinate system where the first base starts at position~0, and the start of the interval is included but the end is not.
  For example, for a sequence of bases~\texttt{ACTGCG}, the bases given by the interval~[2,~4) are~\texttt{TG}. % chktex 9

\item[BED$n$:]
  A~\textbf{file} with the first $n$~\textbf{field}s of the \ac{BED} format.
  For example, \textbf{BED3}~means a~\textbf{file} with only the first 3~\textbf{field}s; \textbf{BED12}~means a~\textbf{file} with all 12~\textbf{field}s.

\item[BED$n$+:]
  A~\textbf{file} that has $n$~\textbf{field}s of the \ac{BED} format, followed by any number of~\textbf{field}s of custom data defined by a user.

\item[BED$n$+$m$:]
  A~\textbf{file} that has a custom tab-delimited format starting with the first $n$~\textbf{field}s of the \ac{BED} format, followed by $m$~\textbf{field}s of custom data defined by a user.
  For example, \textbf{BED6+4}~means a~\textbf{file} with the first 6~\textbf{field}s of the \ac{BED} format, followed by 4~user-defined~\textbf{field}s.

\item[block:]
  Linear subfeatures within a~\textbf{feature}.
  Usually used to designate exons.

\item[chromosome:]
  A sequence of nucleobases with a name.
  In this specification, ``chromosome'' may also describe a named scaffold that does not fit the biological definition of a chromosome.
  Often, chromosomes are numbered starting from~\texttt{1}.
  There are also often sex chromosomes such as~\texttt{W}, \texttt{X}, \texttt{Y}, and~\texttt{Z}, mitochondrial chromosomes such as~\texttt{M}, and possibly scaffolds from an unknown chromosome, often labeled~\texttt{Un}.
  The name of each chromosome is often prefixed with~\texttt{chr}.
  Examples of chromosome names include~\texttt{chr1}, \texttt{21}, \texttt{chrX}, \texttt{chrM}, \texttt{chrUn}, \texttt{chr19\_KI270914v1\_alt}, and~\texttt{chrUn\_KI270435v1}.

\item[feature:]
  A linear region of a~\textbf{chromosome} with specified properties.
  For example, a~\textbf{file}'s~\textbf{feature}s might all be peaks called from ChIP-seq data, or transcript.

\item[field:]
  Data stored as non-tab text.
  All~\textbf{field}s are 7-bit US \ac{ASCII}.

\item[file:]
  Sequence of one or more~\textbf{line}s.

\item[line:]
  String terminated by a~\textbf{line separator}, in one of the following classes.
  Either a~\textbf{data line}, a~\textbf{comment line}, or a~\textbf{blank line}.
  Discussed more fully in~\autoref{sec:lines}

\item[line separator:]
  Either carriage return, line feed, or carriage return followed by line feed.
  The same \textbf{line separator} must be used throughout the \textbf{file}.
\end{description}

\subsection{Lines}\label{sec:lines}

\subsubsection{Data lines}

Data lines contain \textbf{feature}~information.
A data line is composed of~\textbf{field}s separated by whitespace.
The whitespace must match the \ac{regex}~\texttt{[[:space:]]+}\footnote{\texttt{[[:space:]]} includes the following characters: space, form-feed, newline, carriage-return, tab, and vertical-tab}.

\subsubsection{Comment lines and blank lines}

Both comment lines and blank lines provide no~\textbf{feature} data.

Comment lines start with~\texttt{\#} with no whitespace beforehand.
A~\texttt{\#} appearing anywhere else in a line is treated as~\textbf{feature} data, not a comment.

Blank lines consist entirely of whitespace.
Both comment and blank lines may appear as any line in a~\textbf{file}, at the beginning, middle, or end of the file.
They may appear in any quantity.

\subsection{\acs{BED} fields}

Each~\textbf{data line} contains between~3 and 12~whitespace-delimited~\textbf{field}s.
The first 3~\textbf{field}s are mandatory, and the last 9~\textbf{field}s are optional.
In optional~\textbf{field}s, the order is binding---if 1~\textbf{field} is filled, then all previous~\textbf{field}s must also be filled.
However, \textbf{BED10} and \textbf{BED11} are prohibited.

In a \ac{BED}~\textbf{file}, each~\textbf{data line} must have the same number of~\textbf{field}s.
The positions in \ac{BED}~\textbf{field}s are all described in the~\textbf{0-based, half-open coordinate system}.

\begin{adjustwidth}{-0.5in}{-0.5in}
  \noindent
  \begin{tabularx}{\linewidth}{r l l l L}
    \toprule
    Col & Field & Type & Regex or range & Brief description \\
    \midrule
    1 & \textsf{chrom} & String & \texttt{[[:alnum:]\_]\{1,255\}}{\footnotemark} & \textbf{Chromosome} name \\
    2 & \textsf{chromStart} & Int & $[0, 2^{64}-1]$ & \textbf{Feature} start position \\
    3 & \textsf{chromEnd} & Int & $[0, 2^{64} -1]$ & \textbf{Feature} end position \\
    4 & \textsf{name} & String & \texttt{[{\textasciicircum}{\textbackslash}t]\{0,255\}} & \textbf{Feature} description \\
    5 & \textsf{score} & Int & $[0, 1000]$ & A numerical value \\
    6 & \textsf{strand} & String & \texttt{[-+.]} & \textbf{Feature} strand \\
    7 & \textsf{thickStart} & Int & $[0, 2^{64}-1]$ & Thick start position \\
    8 & \textsf{thickEnd} & Int & $[0, 2^{64}-1]$ & Thick end position \\
    9 & \textsf{itemRgb} & Int,Int,Int & \texttt{(}$[0, 255], [0,255], [0,255]$\texttt{) | 0} & Display color \\ % chktex 9
    10 & \textsf{blockCount} & Int & $[0, \textsf{chromEnd}-\textsf{chromStart}]${\footnotemark} & Number of \textbf{block}s \\
    11 & \textsf{blockSizes} & List[Int] & \texttt{([[:digit:]]+,)\{\textsf{blockCount}$-1$\}[[:digit:]]+,?}{\footnotemark} & \textbf{Block} sizes \\
    12 & \textsf{blockStarts} & List[Int] & \texttt{([[:digit:]]+,)\{\textsf{blockCount}$-1$\}[[:digit:]]+,?} & \textbf{Block} start positions \\
    \bottomrule
  \end{tabularx}\label{sec:table}
  \footnotetext[5]{\texttt{[[:alnum:]\_]} is equivalent to the \ac{regex} \texttt{[A-Za-z0-9\_]}. % chktex 8
      It is also equivalent to the Perl extension \texttt{[[:word:]]}.}
  \footnotetext[6]{\textsf{chromEnd}-\textsf{chromStart} is the maximum number of~\textbf{block}s that may exist without overlaps.}
  \footnotetext{For example, if~$\textsf{blockCount} = 4$, then the allowed \ac{regex} would be~\texttt{([[:digit:]]+,)\{3\}[[:digit:]]+,?}}
\end{adjustwidth}

\subsection{Coordinates}
\begin{enumerate}
\item \textsf{chrom}: The name of the~\textbf{chromosome} or scaffold where the~\textbf{feature} is present.
  Limiting only to word characters only, instead of all non-whitespace characters, makes \ac{BED}~\textbf{file}s more portable to varying environments which may make different assumptions about allowed characters.
  The name must be between~1 and 255~characters long, inclusive.

\item \textsf{chromStart}: Start position of the~\textbf{feature} on the~\textbf{chromosome} or scaffold.
  \textsf{chromStart}~must be an integer greater than or equal to~0 and less than the total number of bases of the~\textbf{chromosome} to which it belongs.
  If the size of the~\textbf{chromosome} is unknown, then \textsf{chromStart}~must be less than or equal to~$2^{64} - 1$, which is the maximum size of an unsigned 64-bit integer.

\item \textsf{chromEnd}: End position of the~\textbf{feature} on the~\textbf{chromosome} or scaffold.
  \textsf{chromEnd}~must be an integer greater than or equal to the value of~\textsf{chromStart} and less than or equal to the total number of bases in the~\textbf{chromosome} to which it belongs.
  If the size of the~\textbf{chromosome} is unknown, then \textsf{chromEnd}~must be less than or equal to~$2^{64} - 1$, the maximum size of an unsigned 64-bit integer.
\end{enumerate}

\subsection{Simple attributes}
\begin{enumerate}
  \setcounter{enumi}{3}

\item \textsf{name}: String that describes the~\textbf{feature}.
  The name must be~0 to 255~non-tab characters.
  The name must not be empty or contain whitespace, unless all fields in file are delimited exclusively using single tab characters.
  A visual representation of the \ac{BED} format may display the name next to the~\textbf{feature}.

\item \textsf{score}: Integer between~0 and~1000, inclusive.
  If the~\textbf{feature} has no score information, then~\texttt{0} should be used as a default value.
  A visual representation of the \ac{BED} format may shade features differently depending on their score.

\item \textsf{strand}: Strand that the~\textbf{feature} appears on.
  The strand may either refer to the~\texttt{+}~(sense or coding) strand or the~\texttt{-}~(antisense or complementary) strand.
  If the~\textbf{feature} has no strand information or unknown strand, then a dot~(\texttt{.}) must be used.
\end{enumerate}

\subsection{Display attributes}
\begin{enumerate}
  \setcounter{enumi}{6}

\item \textsf{thickStart}: Start position at which the~\textbf{feature} is visualized with a thicker or accented display.
  This value must be an integer between~\textsf{chromStart} and~\textsf{chromEnd}, inclusive.
  There is no specified default value for~\textsf{thickStart}.

\item \textsf{thickEnd}: End position at which the~\textbf{feature} is visualized with a thicker or accented display.
  This value must be an integer greater than or equal to~\textsf{thickStart} and less than or equal to~\textsf{chromEnd}, inclusive.
  In \ac{BED} files with fewer than 7~\textbf{field}s, the whole~\textbf{feature} has thick display.
  In \textbf{BED7+}~files, to achieve the same effect, set \textsf{thickStart}~equal to~\textsf{chromStart} and \textsf{thickEnd}~equal to~\textsf{chromEnd}.
  If this~\textbf{field} is not specified but \textsf{thickStart}~is, then the entire~\textbf{feature} has thick display.
  There is no specified default value for~\textsf{thickEnd}.

\item \textsf{itemRgb}: A triple of integers that determines the color of this~\textbf{feature} when visualized.
  The triple is three integers separated by commas.
  Each integer is between~0 and~255, inclusive.
  To make a~\textbf{feature} black, \textsf{itemRgb}~may be a single~\texttt{0}, which is visualized identically to a~\textbf{feature} with \textsf{itemRgb} of \texttt{0,0,0}.
\end{enumerate}

\subsection{Blocks}
\begin{enumerate}
  \setcounter{enumi}{9}

\item \textsf{blockCount}: Number of~\textbf{block}s in the~\textbf{feature}.
  \textsf{blockCount}~must be an integer greater than 0.
  \textsf{blockCount}~is mandatory in~\textbf{BED12+}~files.
  Null or empty~\textsf{blockCount} are not allowed, because~\textsf{blockSizes} and~\textsf{blockStarts} rely on~\textsf{blockCount}.
  A visual representation of the \ac{BED} format may have blocks appear thicker than the rest of the~\textbf{feature}.

\item \textsf{blockSizes}: Comma-separated list of length~\textsf{blockCount} containing the size of each~\textbf{block}.
  There must be no spaces before or after commas.
  There may be a trailing comma after the last element of the list.
  \textsf{blockSizes}~is mandatory in \textbf{BED12+} files.
  Null or empty~\textsf{blockSizes} is not allowed, because \textsf{blockStarts}~cannot be verified without~\textsf{blockSizes}.

\item \textsf{blockStarts}: Comma-separated list of length~\textsf{blockCount} containing each \textbf{block}'s~start position, relative to~\textsf{chromStart}.
  There must not be spaces before or after the commas.
  There may be a trailing comma after the last element of the list.
  Each element in~\textsf{blockStarts} is paired with the corresponding element in~\textsf{blockSizes}.
  Each \textsf{blockStarts}~element must be an integer between~0 and~$\textsf{chromEnd} - \textsf{chromStart}$, inclusive.
  For each couple~$i$ of~$(\textsf{blockStarts}_i, \textsf{blockSizes}_i)$, the quantity~$\textsf{chromStart} + \textsf{blockStarts}_i + \textsf{blockSizes}_i$ must be less or equal to \textsf{chromEnd}.
  These conditions enforce that each~\textbf{block} is contained within the~\textbf{feature}.
  The first~\textbf{block} must start at~\textsf{chromStart} and the last~\textbf{block} must end at~\textsf{chromEnd}.
  Moreover, the~\textbf{block}s must not overlap.
  The list must be sorted in ascending order.
  \textsf{blockStarts}~is mandatory in~\textbf{BED12+} files.
  Null or empty~\textsf{blockStarts} is not allowed.
\end{enumerate}

\section{Examples}

\subsection[title]{Example BED6 file from the \acs{UCSC} Genome Browser FAQ\footnote{``Frequently
    Asked Questions: Data File Formats.'' \ac{UCSC} Genome Browser FAQ,
    \url{https://genome.ucsc.edu/FAQ/FAQformat.html}}}\label{sec:example-bed6}

\begin{verbatim}
chr7  127471196  127472363  Pos1  0  +
chr7  127472363  127473530  Pos2  0  +
chr7  127473530  127474697  Pos3  0  +
chr7  127474697  127475864  Pos4  0  +
chr7  127475864  127477031  Neg1  0  -
chr7  127477031  127478198  Neg2  0  -
chr7  127478198  127479365  Neg3  0  -
chr7  127479365  127480532  Pos5  0  +
chr7  127480532  127481699  Neg4  0  -
\end{verbatim}

\subsection{Example BED12 file from the \acs{UCSC} Genome Browser FAQ}
\begin{verbatim}
chr22 1000 5000 cloneA 960 + 1000 5000 0 2 567,488, 0,3512
chr22 2000 6000 cloneB 900 - 2000 6000 0 2 433,399, 0,3601
\end{verbatim}

The~\textbf{block}s in this example satisfy the required constraints.
The first~\textbf{block} starts at~\textsf{chromStart} since the first~\textsf{blockStarts} element is~0.
The last~\textbf{block} ends at~\textsf{chromEnd} since the last~\textbf{block} starts at position 4512~(1000+3512) with size~488, and therefore ends at position 5000~(4512+488).

\section{Recommended practice for the \acs{BED} format}

\subsection{Mandatory fields}
\begin{itemize}
\item \textsf{chrom}: The name of each~\textbf{chromosome} should also match the names from a reference genome, if applicable.
  For example, in the human genome, the chromosomes may be named~\texttt{chr1} to \texttt{chr22}, \texttt{chrX}, \texttt{chrY}, and~\texttt{chrM}.
  Names should be consistent within a~\textbf{file}.
  For example, one should not use both~\texttt{17} and~\texttt{chr17} to represent the same~\textbf{chromosome} in the same~\textbf{file}.
\end{itemize}

\subsection{Optional fields}\label{sec:optional}
\begin{itemize}
\item \textsf{name}: If a feature has no name, then a dot~(\texttt{.}) should be used.
  Names should avoid using the space character even if the file is exclusively delimited with single tab characters because parsers may interpret a space as a delimiter.

\item \textsf{itemRgb}: Eight or fewer colors should be used as too many colors may slow down visualizations and are difficult for humans to distinguish.\footnote{``Frequently
    Asked Questions: Data File Formats.'' \ac{UCSC} Genome Browser FAQ,
    \url{https://genome.ucsc.edu/FAQ/FAQformat.html}}

\end{itemize}

\subsection{User-defined fields}

Custom data \textbf{fields} may contain any non-tab 7-bit US \ac{ASCII} character.
Definitions of a custom \ac{BED} format should restrict the type of each \textbf{field} to the extent possible.
Each custom \textbf{field} should contain either one of the following data types or a comma-separated list of values of the same type:

\noindent
\begin{tabularx}{\textwidth}{r L}
  \toprule
  Type & Definition \\
  \midrule
  Integer & String representation of 64-bit signed integer\footnote{\emph{IEEE 754--1985 IEEE Standard for Binary Floating-Point Arithmetic.} IEEE 754--1985, 1985} \\
  Unsigned & String representation of 64-bit unsigned integer\footnotemark[10] \\
  Float & String representation of 64-bit floating point number\footnotemark[10] \\
  Character & One character, other than tab \\
  String & One or more characters, other than tab \\
  \bottomrule
\end{tabularx}

This specification does not contain a means for interchanging custom \ac{BED} format definitions.
The AutoSQL format\footnote{Kent, W.~James.
  (2000) ``AutoSQL.''
  \url{https://hgwdev.gi.ucsc.edu/~kent/exe/doc/autoSql.doc}} provides one method for defining custom \ac{BED} formats in a separate file.

\subsection{Sorting}
\Ac{BED} \textbf{file}s should be sorted by~\textsf{chrom}, then by~\textsf{chromStart} numerically, and finally by~\textsf{chromEnd} numerically.
\textsf{chrom} may be sorted using any scheme (such as lexicographic or numeric order), but all lines with the same~\textsf{chrom} value should occur consecutively.
For example, the lexicographic order of~\texttt{chr1}, \texttt{chr10}, \texttt{chr11}, \texttt{chr12}, {\ldots}, \texttt{chr2}, \texttt{chr20}, \texttt{chr21}, {\ldots}, \texttt{chr3}, {\ldots}, \texttt{chrX}, \texttt{chrY}, \texttt{chrM} is an acceptable sorting.
The numeric order of~\texttt{chr1}, \texttt{chr2}, {\ldots}, \texttt{chr21}, \texttt{chr22}, \texttt{chrM}, \texttt{chrX}, \texttt{chrY} is also acceptable.
Regardless of the chromosome sorting scheme, lines for two features on the same chromosome should not have any lines for features on other chromosomes between them.

\subsection{Whitespace}\label{sec:whitespace}
Though lines may use any kind of whitespace as a delimiter between~\textbf{field}s, a single tab~(\texttt{{\textbackslash}t}) should be used.
This is because almost all tools support tabs while some tools do not support other kinds of whitespace.
Also, whitespace within the~\textsf{name}~\textbf{field} may be used only if the \textbf{field}~delimiter is tab throughout the \textbf{file}.

\subsection{Large \acs{BED} files}
If a~\textbf{file} intended for visualization is over \SI{50}{\mebi\byte} in size, the~\textbf{file} should be converted to~\texttt{bigBed} format, which is an indexed binary format.\footnote{Kent, W.~James et al.
  (2010) ``BigWig and BigBed: enabling browsing of large distributed datasets.''
  \emph{Bioinformatics} 26(17):2204--2207.
  \url{https://doi.org/10.1093/bioinformatics/btq351}}
The~\texttt{bedToBigBed} program may perform this conversion.\footnote{``bigBed Track Format.''
  \Ac{UCSC} Genome Browser FAQ, \url{https://genome.ucsc.edu/goldenPath/help/bigBed.html}}

\section{\acs{UCSC} track files}

Track files are files that contain additional information intended for a visualization tool such as the \ac{UCSC} Genome Browser.\footnote{Haeussler, Maximilian et al.
  (2019) ``The \acl{UCSC} Genome Browser database: 2019 update.''
  \emph{Nucleic Acids Research} 47(D1):D853--D858.
  \url{https://doi.org/10.1093/nar/gky1095}}
Track files contain browser lines and track lines that precede lines from a file format supported by the Genome Browser.\footnote{``Displaying your own annotations in the Genome Browser.'' \ac{UCSC} Genome Browser FAQ, \url{https://genome.ucsc.edu/goldenPath/help/customTrack.html\#lines}}
Track files are not valid \ac{BED} files --- valid \ac{BED} files must not have any browser or track lines.
To distinguish between \ac{BED} files and track files, track files should use the file extension~\texttt{.track}.

\section{Acronyms}
\begin{acronym}[ASCII]
  \acro{ASCII}{American Standard Code for Information Interchange}
  \acro{BED}{Browser Extensible Data}
  \acro{GA4GH}{Global Alliance for Genomics and Health}
  \acro{regex}{regular expression}
  \acro{UCSC}{University of California, Santa Cruz}
\end{acronym}

\section{Acknowledgments}

We thank W.~James Kent and the \ac{UCSC} Genome Browser team for creating the \ac{BED} format.
We thank W.~James Kent and Hiram Clawson~(\ac{UCSC}); Eric Roberts~(Princess Margaret Cancer Centre); John Marshall~(University of Glasgow); Aaron R.~Quinlan and Brent S.~Pedersen~(University of Utah); Ting Wang~(Washington University in St.~Louis); and the \ac{GA4GH} File Formats Task Team for comments on this specification.

\end{document}

% chktex-file 17

%%% Local Variables:
%%% mode: latex
%%% TeX-master: t
%%% End:


\hypersetup{colorlinks=true,
  linkcolor=blue,
  filecolor=magenta,
  urlcolor=blue,
  pdfinfo={githash=\commitdesc}}

\definecolor{cverbbg}{gray}{0.93}

\title{The \acf{BED} format}
\author{Jeffrey Niu, Danielle Denisko, Michael M.~Hoffman}
\date{\headdate}

\setlength{\emergencystretch}{\hsize}
\setlength{\footnotemargin}{1em}

\floatplacement{table}{htbp}
\setcounter{topnumber}{2}
\setcounter{bottomnumber}{2}
\setcounter{totalnumber}{4}
\setcounter{dbltopnumber}{2}
\renewcommand{\dbltopfraction}{0.9}
\renewcommand{\textfraction}{0.07}
\renewcommand{\floatpagefraction}{0.7}

\interfootnotelinepenalty=1000000
\makesavenoteenv{tabularx}

\newcolumntype{L}{>{\raggedright\arraybackslash}X}

\providecommand*{\Ac}[1]{\ac{#1}} % work around outdated acronym.sty packages
\newcommand*{\acrodefused}[2]{\acrodef{#1}{#2}\acused{#1}}

\frenchspacing

% eliminate passive voice warnings
% chktex-file -3

\begin{document}

\maketitle

\begin{small}
\noindent
The master version of this document can be found at \url{https://github.com/samtools/hts-specs}.
This printing is version~\commitdesc\ from that repository, last modified on the date shown above.
\end{small}

\acused{ASCII}

\section{Specification}

\Ac{BED} is a whitespace-delimited file format, where each~\textbf{file} consists of zero or more~\textbf{line}s.\footnote{``Frequently Asked Questions: Data File Formats.'' \ac{UCSC} Genome Browser FAQ, \url{https://genome.ucsc.edu/FAQ/FAQformat.html}}
Data are in~\textbf{data line}s, which describe discrete genomic~\textbf{feature}s by physical start and end position on a linear~\textbf{chromosome}.
The file extension for the \ac{BED} format is~\texttt{.bed}.

\subsection{Scope}

This specification formalizes reasonable interpretations of the \ac{UCSC} Genome Browser \ac{BED} description.
This specification also makes clear potential interoperability issues in the current format, which could be addressed in a future specification.

\subsection{Typographic conventions}

This document uses several typographic conventions~(\autoref{tab:typographic-conventions}).

\begin{savenotes}
  \begin{table}
    \begin{tabularx}{\textwidth}{r L L}
      \toprule
      Style & Meaning & Examples \\
      \midrule
      Bold & Terms defined in subsections~\ref{sec:terms}--\ref{sec:lines} & \textbf{chromosome}{\quad}\textbf{file} \\
      Sans serif & Names of~\textbf{field}s & \textsf{chrom}{\quad}\textsf{chromStart}{\quad}\textsf{chromEnd} \\
      Fixed-width & Literals or \ac{regex}es\footnote{POSIX/IEEE~1003.1--2017 Extended Regular Expressions, for the ``C'' locale.
                    \emph{IEEE Standard for Information Technology---Portable Operating System Interface~(POSIX) Base Specifications}, IEEE~1003.1--2017, 2017} & \texttt{.bed}{\quad}\texttt{grep}{\quad}\texttt{[[:alnum:]]+}{\quad}\texttt{ATCG} \\
      \bottomrule
    \end{tabularx}
    \caption{\textbf{Typographic conventions.}}\label{tab:typographic-conventions}
  \end{table}
\end{savenotes}

\subsection{Terminology and concepts}\label{sec:terms}
\begin{description}
\item[0-based, half-open coordinate system:]
  A coordinate system where the first base starts at position~0, and the start of the interval is included but the end is not.
  For example, for a sequence of bases~\texttt{ACTGCG}, the bases given by the interval~[2,~4) are~\texttt{TG}. % chktex 9

\item[\acs{BED}$n$:]
  A~\textbf{file} with the first $n$~\textbf{field}s of the \ac{BED} format.
  For example, \textbf{BED3}~means a~\textbf{file} with only the first 3~\textbf{field}s; \textbf{BED12}~means a~\textbf{file} with all 12~\textbf{field}s.

\item[\acs{BED}$n$+:]
  A~\textbf{file} that has the first $n$~\textbf{field}s of the \ac{BED} format, followed by any number of~\textbf{field}s of custom data defined by a user.

\item[\acs{BED}$n$+$m$:]
  A~\textbf{file} that has a custom format starting with the first $n$~\textbf{field}s of the \ac{BED} format, followed by $m$~\textbf{field}s of custom data defined by a user.
  For example, \textbf{BED6+4}~means a~\textbf{file} with the first 6~\textbf{field}s of the \ac{BED} format, followed by 4~user-defined~\textbf{field}s.

\item[block:]
  Linear subfeatures within a~\textbf{feature}.
  Usually used to designate exons.

\item[chromosome:]
  A sequence of nucleobases with a name.
  In this specification, ``chromosome'' may also describe a named scaffold that does not fit the biological definition of a chromosome.
  Often, \textbf{chromosome}s are numbered starting from~\texttt{1}.
  There are also often sex \textbf{chromosome}s such as~\texttt{W}, \texttt{X}, \texttt{Y}, and~\texttt{Z}, mitochondrial \textbf{chromosome}s such as~\texttt{M}, and possibly scaffolds from an unknown chromosome, often labeled~\texttt{Un}.
  The name of each \textbf{chromosome} is often prefixed with~\texttt{chr}.
  Examples of \textbf{chromosome} names include~\texttt{chr1}, \texttt{21}, \texttt{chrX}, \texttt{chrM}, \texttt{chrUn}, \texttt{chr19\_KI270914v1\_alt}, and~\texttt{chrUn\_KI270435v1}.

\item[feature:]
  A linear region of a~\textbf{chromosome} with specified properties.
  For example, a~\textbf{file}'s~\textbf{feature}s might all be peaks called from ChIP-seq data, or transcript.

\item[field:]
  Data stored as non-tab text.
  All~\textbf{field}s are 7-bit US \ac{ASCII} printable characters\footnote{Characters in the range \texttt{{\textbackslash}x20} to \texttt{{\textbackslash}x7e}, therefore not including any control characters}.
  Only some \textbf{field}s can be empty, and they can only be empty when a single tab is used as the \textbf{field separator}.

\item[field separator:]
  One or more horizontal whitespace characters (space or tab).
  The \textbf{field} separator must match the \ac{regex}~\texttt{[ {\textbackslash}t]+}.
  The \textbf{field} separator can vary throughout the \textbf{file}.
  Some capabilities of the \ac{BED} format, however, are available only when a single tab is used as the \textbf{field} separator throughout the \textbf{file}.

\item[file:]
  Sequence of one or more~\textbf{line}s.

\item[line:]
  String terminated by a~\textbf{line separator}, in one of the following classes.
  Either a~\textbf{data line}, a~\textbf{comment line}, or a~\textbf{blank line}.
  Discussed more fully in~\autoref{sec:lines}

\item[line separator:]
  Either carriage return~(\texttt{{\textbackslash}r}, equivalent to \texttt{{\textbackslash}x0d}), newline~(\texttt{{\textbackslash}r}, equivalent to \texttt{{\textbackslash}x0a}), or carriage return followed by newline~(\texttt{{\textbackslash}r{\textbackslash}n}, equivalent to \texttt{{\textbackslash}x0d{\textbackslash}x0a}).
  The same \textbf{line separator} must be used throughout the \textbf{file}.
\end{description}

\subsection{Lines}\label{sec:lines}

\subsubsection{Data lines}

\textbf{Data line}s contain \textbf{feature}~information.
A \textbf{data line} is composed of~\textbf{field}s separated by \textbf{field separator}s.

\subsubsection{Comment lines and blank lines}

Both \textbf{comment line}s and \textbf{blank line}s provide no~\textbf{feature} data.

\textbf{Comment line}s start with~\texttt{\#} with no horizontal whitespace beforehand.
A~\texttt{\#} appearing anywhere else in a \textbf{data line} is treated as~\textbf{feature} data, not a comment.

\textbf{Blank line}s consist entirely of horizontal whitespace.
Both comment and blank \textbf{line}s may appear as any \textbf{line} in a~\textbf{file}, at the beginning, middle, or end of the \textbf{file}.
They may appear in any quantity.

\subsection{\acs{BED} fields}

Each~\textbf{data line} contains between~3 and 12~~\textbf{field}s delimited by a \textbf{field separator}.
The first 3~\textbf{field}s are mandatory, and the last 9~\textbf{field}s are optional~(\autoref{tab:fields}).
In optional~\textbf{field}s, the order is binding---if a \textbf{field} is filled, then all previous~\textbf{field}s must also be filled.
However, \textbf{BED10} and \textbf{BED11} are prohibited.

\begin{savenotes}
  \begin{table}
    \begin{adjustwidth}{-0.5in}{-0.5in}
      \begin{tabularx}{\linewidth}{r l l l L}
        \toprule
        Col & Field & Type & Regex or range & Brief description \\
        \midrule
        1
        & \textsf{chrom}
        & String
        & \texttt{[[:alnum:]\_]\{1,255\}}\footnote{\texttt{[[:alnum:]\_]} is equivalent to the \ac{regex} \texttt{[A-Za-z0-9\_]}. % chktex 8
        It is also equivalent to the Perl extension \texttt{[[:word:]]}}
        & \textbf{Chromosome} name \\

        2 & \textsf{chromStart} & Int & $[0, 2^{64}-1]$ & \textbf{Feature} start position \\
        3 & \textsf{chromEnd} & Int & $[0, 2^{64} -1]$ & \textbf{Feature} end position \\
        4 & \textsf{name} & String & \texttt{[{\textbackslash}x20-{\textbackslash}x7e]\{1,255\}} & \textbf{Feature} description \\
        5 & \textsf{score} & Int & $[0, 1000]$ & A numerical value \\
        6 & \textsf{strand} & String & \texttt{[-+.]} & \textbf{Feature} strand \\
        7 & \textsf{thickStart} & Int & $[0, 2^{64}-1]$ & Thick start position \\
        8 & \textsf{thickEnd} & Int & $[0, 2^{64}-1]$ & Thick end position \\
        9 & \textsf{itemRgb} & Int,Int,Int & \texttt{(}$[0, 255], [0,255], [0,255]$\texttt{) | 0} & Display color \\ % chktex 9

        10
        & \textsf{blockCount}
        & Int
        & $[0, \textsf{chromEnd}-\textsf{chromStart}]$\footnote{\textsf{chromEnd}-\textsf{chromStart} is the maximum number of~\textbf{block}s that may exist without overlaps}
        & Number of \textbf{block}s \\

        11
        & \textsf{blockSizes}
        & List[Int]
        & \texttt{([[:digit:]]+,)\{\textsf{blockCount}$-1$\}[[:digit:]]+,?}\footnote{For example, if~$\textsf{blockCount} = 4$, then the allowed \ac{regex} would be~\texttt{([[:digit:]]+,)\{3\}[[:digit:]]+,?}}
        & \textbf{Block} sizes \\

        12 & \textsf{blockStarts} & List[Int] & \texttt{([[:digit:]]+,)\{\textsf{blockCount}$-1$\}[[:digit:]]+,?} & \textbf{Block} start positions \\
        \bottomrule
      \end{tabularx}
    \end{adjustwidth}
    \caption{\textbf{Fields.}}\label{tab:fields}
  \end{table}
\end{savenotes}

In a \ac{BED}~\textbf{file}, each~\textbf{data line} must have the same number of~\textbf{field}s.
The positions in \ac{BED}~\textbf{field}s are all described in the~\textbf{0-based, half-open coordinate system}.

\subsection{Coordinates}
\begin{enumerate}
\item \textsf{chrom}: The name of the~\textbf{chromosome} where the~\textbf{feature} is present.
  Limiting to word characters only, instead of all non-whitespace printable characters, makes \ac{BED}~\textbf{file}s more portable to varying environments which may make different assumptions about allowed characters.
  The name must be between~1 and 255~characters long, inclusive.

\item \textsf{chromStart}: Start position of the~\textbf{feature} on the~\textbf{chromosome}.
  \textsf{chromStart}~must be an integer greater than or equal to~0 and less than or equal to the total number of bases of the~\textbf{chromosome} to which it belongs.
  If the size of the~\textbf{chromosome} is unknown, then \textsf{chromStart}~must be less than or equal to~$2^{64} - 1$, which is the maximum size of an unsigned 64-bit integer.

\item \textsf{chromEnd}: End position of the~\textbf{feature} on the~\textbf{chromosome}.
  \textsf{chromEnd}~must be an integer greater than or equal to the value of~\textsf{chromStart} and less than or equal to the total number of bases in the~\textbf{chromosome} to which it belongs.
  If \textsf{chromEnd}~is equal to~\textsf{chromStart}, this indicates a \textbf{feature} between \textsf{chromStart} and the preceding base, such as an insertion.
  When \textsf{chromStart} and \textsf{chromEnd} are both~0, this indicates a feature before the entire~\textbf{chromosome}.
  If the size of the~\textbf{chromosome} is unknown, then \textsf{chromEnd}~must be less than or equal to~$2^{64} - 1$, the maximum size of an unsigned 64-bit integer.
\end{enumerate}

\subsection{Simple attributes}
\begin{enumerate}
  \setcounter{enumi}{3}

\item \textsf{name}: String that describes the~\textbf{feature}.
  \textsf{name} must be~1 to 255~non-tab characters.
  \textsf{name} must not be empty.
  \textsf{name} must not contain whitespace, unless the only \textbf{field separator} is a single tab.
  Multiple \textbf{data line}s may share the same \textsf{name}.
  In \textbf{BED5+} \textbf{file}s where all \textbf{features} have uninformative \textsf{name}s, dot~(\texttt{.}) may be used as a \textsf{name} on every \textbf{data line}.
  A visual representation of the \ac{BED} format may display \textsf{name} next to the~\textbf{feature}.

\item \textsf{score}: Integer between~0 and~1000, inclusive.
  In \textbf{BED6+} \textbf{file}s where all \textbf{feature}s have uninformative \textsf{score}s, \texttt{0} should be used as the \textsf{score} on every \textbf{data line}.
  A visual representation of the \ac{BED} format may shade \textbf{feature}s differently depending on their \textsf{score}.

\item \textsf{strand}: Strand that the~\textbf{feature} appears on.
  The \textsf{strand} may either refer to the~\texttt{+}~(sense or coding) strand or the~\texttt{-}~(antisense or complementary) strand.
  If the \textbf{feature} has no \textsf{strand} information or unknown \textsf{strand}, then a dot~(\texttt{.}) must be used as an uninformative value.
  \textsf{strand} cannot be empty in \textbf{BED6+} \textbf{file}s.
  \textsf{strand} should be treated as \texttt{.} when parsing files that are not \textbf{BED6+}.
\end{enumerate}

\subsection{Display attributes}
\begin{enumerate}
  \setcounter{enumi}{6}

\item \textsf{thickStart}: Start position at which the~\textbf{feature} is visualized with a thicker or accented display.
  This value must be an integer between~\textsf{chromStart} and~\textsf{chromEnd}, inclusive.
  In \textbf{BED7+} \textbf{file}s where all \textbf{feature}s have uninformative \textsf{thickStart}s, the value of \textsf{chromStart} should be used as the \textsf{thickStart} on every \textbf{data line}.

\item \textsf{thickEnd}: End position at which the~\textbf{feature} is visualized with a thicker or accented display.
  This value must be an integer greater than or equal to~\textsf{thickStart} and less than or equal to~\textsf{chromEnd}, inclusive.
  In \textbf{BED8+} \textbf{file}s where all \textbf{feature}s have uninformative \textsf{thickEnd}s, the value of \textsf{chromEnd} should be used as the \textsf{thickEnd} on every \textbf{data line}.
  In \ac{BED} \textbf{file}s that are not \textbf{BED7+}, the whole~\textbf{feature} has thick display.
  In \textbf{BED7+}~\textbf{file}s, to achieve the same effect, set \textsf{thickStart}~equal to~\textsf{chromStart} and \textsf{thickEnd}~equal to~\textsf{chromEnd}.
  If this~\textbf{field} is not specified but \textsf{thickStart}~is, then the entire~\textbf{feature} has thick display.

\item \textsf{itemRgb}: A triple of integers that determines the color of this~\textbf{feature} when visualized.
  The triple is three integers separated by commas.
  Each integer is between~0 and~255, inclusive.
  To make a~\textbf{feature} black, \textsf{itemRgb}~may be a single~\texttt{0}, which is visualized identically to a~\textbf{feature} with \textsf{itemRgb} of \texttt{0,0,0}.
  An \textsf{itemRgb} of~\texttt{0} is a special case and no other single-number value is valid.
  In \textbf{BED9+} \textbf{file}s where all \textbf{feature}s have uninformative \textsf{itemRgb}s, \texttt{0} should be used as the \textsf{itemRgb} on every \textbf{data line}.
\end{enumerate}

\subsection{Blocks}
\begin{enumerate}
  \setcounter{enumi}{9}

\item \textsf{blockCount}: Number of~\textbf{block}s in the~\textbf{feature}.
  \textsf{blockCount}~must be an integer greater than 0.
  \textsf{blockCount}~is mandatory in~\textbf{BED12+}~\textbf{file}s.
  Empty~\textsf{blockCount} are not allowed, because~\textsf{blockSizes} and~\textsf{blockStarts} rely on~\textsf{blockCount}.
  A visual representation of the \ac{BED} format may have blocks appear thicker than the rest of the~\textbf{feature}.

\item \textsf{blockSizes}: Comma-separated list of length~\textsf{blockCount} containing the size of each~\textbf{block}.
  There must be no spaces before or after commas.
  There may be a trailing comma after the last element of the list.
  \textsf{blockSizes}~is mandatory in \textbf{BED12+} \textbf{file}s.
  Empty~\textsf{blockSizes} is not allowed, because \textsf{blockStarts}~cannot be verified without~\textsf{blockSizes}.

\item \textsf{blockStarts}: Comma-separated list of length~\textsf{blockCount} containing each \textbf{block}'s~start position, relative to~\textsf{chromStart}.
  There must not be spaces before or after the commas.
  There may be a trailing comma after the last element of the list.
  Each element in~\textsf{blockStarts} is paired with the corresponding element in~\textsf{blockSizes}.
  Each \textsf{blockStarts}~element must be an integer between~0 and~$\textsf{chromEnd} - \textsf{chromStart}$, inclusive.
  For each couple~$i$ of~$(\textsf{blockStarts}_i, \textsf{blockSizes}_i)$, the quantity~$\textsf{chromStart} + \textsf{blockStarts}_i + \textsf{blockSizes}_i$ must be less or equal to \textsf{chromEnd}.
  These conditions enforce that each~\textbf{block} is contained within the~\textbf{feature}.
  The first~\textbf{block} must start at~\textsf{chromStart} and the last~\textbf{block} must end at~\textsf{chromEnd}.
  Moreover, the~\textbf{block}s must not overlap.
  The list must be sorted in ascending order.
  \textsf{blockStarts}~is mandatory in~\textbf{BED12+} \textbf{file}s.
  Empty~\textsf{blockStarts} is not allowed.
\end{enumerate}

\section{Examples}

\subsection[title]{Example BED6 file from the \acs{UCSC} Genome Browser FAQ\footnote{``Frequently
    Asked Questions: Data File Formats.'' \ac{UCSC} Genome Browser FAQ,
    \url{https://genome.ucsc.edu/FAQ/FAQformat.html}}}\label{sec:example-bed6}

\begin{verbatim}
chr7  127471196  127472363  Pos1  0  +
chr7  127472363  127473530  Pos2  0  +
chr7  127473530  127474697  Pos3  0  +
chr7  127474697  127475864  Pos4  0  +
chr7  127475864  127477031  Neg1  0  -
chr7  127477031  127478198  Neg2  0  -
chr7  127478198  127479365  Neg3  0  -
chr7  127479365  127480532  Pos5  0  +
chr7  127480532  127481699  Neg4  0  -
\end{verbatim}

\subsection{Example BED12 file from the \acs{UCSC} Genome Browser FAQ}
\begin{verbatim}
chr22 1000 5000 cloneA 960 + 1000 5000 0 2 567,488, 0,3512
chr22 2000 6000 cloneB 900 - 2000 6000 0 2 433,399, 0,3601
\end{verbatim}

The~\textbf{block}s in this example satisfy the required constraints.
The first~\textbf{block} starts at~\textsf{chromStart} since the first~\textsf{blockStarts} element is~0.
The last~\textbf{block} ends at~\textsf{chromEnd} since the last~\textbf{block} starts at position 4512~(1000+3512) with size~488, and therefore ends at position 5000~(4512+488).

\section{Recommended practice for the \acs{BED} format}

\subsection{Mandatory fields}
\begin{itemize}
\item \textsf{chrom}: The name of each~\textbf{chromosome} should also match the names from a reference genome, if applicable.
  For example, in the human genome, the \textbf{chromosome}s may be named~\texttt{chr1} to \texttt{chr22}, \texttt{chrX}, \texttt{chrY}, and~\texttt{chrM}.
  Names should be consistent within a~\textbf{file}.
  For example, one should not use both~\texttt{17} and~\texttt{chr17} to represent the same~\textbf{chromosome} in the same~\textbf{file}.
\end{itemize}

\subsection{Optional fields}\label{sec:optional}
\begin{itemize}
\item \textsf{name}: In the absence of out-of-band information about whether a single tab is the only \textbf{field separator} in the file, if a \textbf{feature} has no name, then a dot~(\texttt{.}) should be used.
  Names should avoid using the space character even if the only \textbf{field separator} is a single tab character, because parsers may interpret a space as a \textbf{field separator}.

\item \textsf{itemRgb}: Eight or fewer colors should be used as too many colors may slow down visualizations and are difficult for humans to distinguish.\footnote{``Frequently
    Asked Questions: Data File Formats.'' \ac{UCSC} Genome Browser FAQ,
    \url{https://genome.ucsc.edu/FAQ/FAQformat.html}}
  Color schemes should be colorblind-friendly.
  Red-green color schemes should be avoided.

\end{itemize}

\subsection{User-defined fields}

Custom data \textbf{fields} may contain any printable 7-bit US \ac{ASCII} character (which includes spaces, but excludes tabs, newlines, and other control characters).
Custom data \textbf{fields} can only be empty when a single tab is used as the \textbf{field separator} throughout the \textbf{file}.
Definitions of a custom \ac{BED} format should restrict the type of each \textbf{field} to the extent possible.
Each custom \textbf{field} should contain either one of several specified data types~(\autoref{tab:custom-data-types}) or a comma-separated list of values of any type other than String.

\begin{savenotes}
  \begin{table}
    \begin{tabularx}{\textwidth}{r L}
      \toprule
      Type & Definition \\
      \midrule
      Integer & Decimal string representation of 64-bit signed integer \\
      Unsigned & Decimal string representation of 64-bit unsigned integer \\
      Float & Decimal string representation of 64-bit floating point number\footnote{\emph{IEEE Standard for Binary Floating-Point Arithmetic.}
              IEEE 754--1985, 1985} \\
      Character & One character, other than tab \\
      String & One or more characters, other than tab \\
      \bottomrule
    \end{tabularx}
    \caption{\textbf{Custom field data types.}}\label{tab:custom-data-types}
  \end{table}
\end{savenotes}

This specification does not contain a means for interchanging custom \ac{BED} format definitions.
The AutoSQL format\footnote{Kent, W.~James.
  (2000) ``AutoSQL.''
  \url{https://hgwdev.gi.ucsc.edu/~kent/exe/doc/autoSql.doc}} provides one method for defining custom \ac{BED} formats in a separate file.

\subsection{Sorting}
\Ac{BED} \textbf{file}s should be sorted by~\textsf{chrom}, then by~\textsf{chromStart} numerically, and finally by~\textsf{chromEnd} numerically.
\textsf{chrom} may be sorted using any scheme (such as lexicographic or numeric order), but all \textbf{data line}s with the same~\textsf{chrom} value should occur consecutively.
For example, the lexicographic order of~\texttt{chr1}, \texttt{chr10}, \texttt{chr11}, \texttt{chr12}, {\ldots}, \texttt{chr2}, \texttt{chr20}, \texttt{chr21}, {\ldots}, \texttt{chr3}, {\ldots}, \texttt{chrX}, \texttt{chrY}, \texttt{chrM} is an acceptable sorting.
This ordering is equivalent to sorting the \textbf{file} using the command \verb|LC\_ALL=C| \verb|sort| \verb|-k 1,1| \verb|-k 2,2n| \verb|-k 3,3n|.
The numeric order of~\texttt{chr1}, \texttt{chr2}, {\ldots}, \texttt{chr21}, \texttt{chr22}, \texttt{chrM}, \texttt{chrX}, \texttt{chrY} is also acceptable.
Arbitrary orderings of~\textsf{chrom} are allowed, but regardless of the \textbf{chromosome} sorting scheme, \textbf{data line}s for two \textbf{feature}s on the same \textbf{chromosome} should not have any \textbf{data line}s for \textbf{feature}s on other \textbf{chromosome}s between them.
Multiple \textbf{feature}s that have the same~\textsf{chrom}, \textsf{chromStart}, and \textsf{chromEnd} can appear in any order.
\textbf{Comment line}s and \textbf{blank line}s do not have to be sorted according to the schemes mentioned.

Sorting is recommended because the implementation of downstream operations is easier if features of one chromosome are all grouped together and \textsf{chromStart} is non-decreasing within a chromosome.

For \textbf{BED4+} files, a sorting scheme may also order by optional \textbf{field}s and any custom \textbf{field}s.
A recommendation for how to do this is outside the scope of this version of the specification.
Total deterministic sorting of \ac{BED} \textbf{file}s can prevent downstream analyses from producing different results depending on sort order.

\subsection{Whitespace}\label{sec:whitespace}
We recommend that only a single tab~(\texttt{{\textbackslash}t}) be used as \textbf{field separator}.
This is because almost all tools support tabs while some tools do not support other kinds of whitespace.
Also, spaces within the~\textsf{name}~\textbf{field} may be used only if the \textbf{field separator} is tab throughout the \textbf{file}.

It would be sensible for future major versions of this specification or overlay formats built on top of this specification to require that only a single tab be used as \textbf{field separator}.

\subsection{Large \acs{BED} files}
If a~\textbf{file} intended for visualization is over \SI{50}{\mebi\byte} in size, the~\textbf{file} should be converted to~\texttt{bigBed} format, which is an indexed binary format.\footnote{Kent, W.~James et al.
  (2010) ``BigWig and BigBed: enabling browsing of large distributed datasets.''
  \emph{Bioinformatics} 26(17):2204--2207.
  \url{https://doi.org/10.1093/bioinformatics/btq351}}
The~\texttt{bedToBigBed} program may perform this conversion.\footnote{``bigBed Track Format.''
  \Ac{UCSC} Genome Browser FAQ, \url{https://genome.ucsc.edu/goldenPath/help/bigBed.html}}

Tabix is another option for storing larger \ac{BED} \textbf{file}s.\footnote{Li H.
  (2011) ``Tabix: fast retrieval of sequence features from generic TAB-delimited files.''
  \emph{Bioinformatics} 27(5):718--719.
  \url{https://doi.org/10.1093/bioinformatics/btq671}}
Tabix works only on \textbf{file}s using a single tab as the \textbf{field separator}.

\section{Information supplied out-of-band}

Some information about a \ac{BED} \textbf{file} can only be supplied unambiguously separately from the \textbf{data line}s of the \ac{BED} \textbf{file}.
This specification does not contain a means for interchanging this information.
Information that must be supplied out-of-band include:

\begin{itemize}[noitemsep]
    \item Which of the first~4 to 12~\textbf{fields} are standard \ac{BED} \textbf{fields} and which are custom \textbf{field}s.
    \item The genome assembly that defines \textsf{chrom}, \textsf{chromStart}, and \textsf{chromEnd}.
    \item The semantics of \textbf{fields} such as \textsf{score}, \textsf{itemRgb}, thick vs.~thin positions, and block vs.~non-block positions.
    \item The definitions of custom \textbf{field}s.
    \item Whether the \textbf{field separator} is a single tab character.
\end{itemize}

\section{\acs{UCSC} track files}

Track files are files that contain additional information intended for a visualization tool such as the \ac{UCSC} Genome Browser.\footnote{Haeussler, Maximilian et al.
  (2019) ``The \acs{UCSC} Genome Browser database: 2019 update.''
  \emph{Nucleic Acids Research} 47(D1):D853--D858.
  \url{https://doi.org/10.1093/nar/gky1095}}
Track files contain browser lines and track lines that precede lines from a file format supported by the Genome Browser.\footnote{``Displaying your own annotations in the Genome Browser.'' \ac{UCSC} Genome Browser FAQ, \url{https://genome.ucsc.edu/goldenPath/help/customTrack.html\#lines}}
Track files are not valid \ac{BED} \textbf{file}s---valid \ac{BED} \textbf{file}s must not have any browser or track lines.
To distinguish between \ac{BED} \textbf{file}s and track files, track files should use the file extension~\texttt{.track}.

\section{Acronyms}

% using the optional argument to acronym to set the label width causes it to use the list environment instead of description, which means we can't set nosep easily
\setlist[description]{labelwidth=\widthof{\textbf{\acs{GA4GH}}},nosep}
\begin{acronym}
  \acro{ASCII}{American Standard Code for Information Interchange}
  \acro{BED}{Browser Extensible Data}
  \acro{GA4GH}{Global Alliance for Genomics and Health}
  \acro{regex}{regular expression}
  \acro{UCSC}{University of California, Santa Cruz}
\end{acronym}

\acrodefused{EMBL}{European Molecular Biology Laboratory}

\section{Acknowledgments}

We thank W.~James Kent and the \ac{UCSC} Genome Browser team for creating the \ac{BED} format.
We thank W.~James Kent and Hiram Clawson~(\ac{UCSC}); Eric Roberts~(University Health Network); John Marshall~(University of Glasgow); Aaron R.~Quinlan and Brent S.~Pedersen~(University of Utah); Ting Wang~(Washington University in St.~Louis); Daniel Perrett and Simon Brent~(Wellcome Sanger Institute); Jasper Saris~(Erasmus Medical Center); Zhenyu Zhang (University of Chicago); Andrew Yates~(\ac{EMBL}---European Bioinformatics Institute); Michael Schatz~(Johns Hopkins University); Igor Dolgalev (New York University); Colin Diesh~(University of California, Berkeley); Alex Reynolds~(Altius Institute for Biomedical Sciences); Junjun Zhang~(Ontario Institute for Cancer Research); and the \ac{GA4GH} File Formats Task Team for comments on this specification.

\end{document}

% chktex-file 17

%%% Local Variables:
%%% mode: latex
%%% TeX-master: t
%%% End:
